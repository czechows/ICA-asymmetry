

\subsection{Measure of nongaussianity}

We consider the similar idea to the Kullback-Leibler.

\subsection{Construction of densities}

We can define the family of singular densities on affine subspaces of 
dimension $d$, by taking the transport.


In this subsection we describe the basic construction of product measures and densities. Given functions $f_1,f_2$ on $\R^{d_1},\R^{d_2}$ by 
$$
(f_1 \otimes f_2)(x_1,x_2)=f_1(x_1) \cdot f_2(x_2) \for (x_1,x_2) \in \R^{d_1} 
\times \R^{d_2}
$$ 
we denote the tensor
product of $f_1$ and $f_2$. Observe that if $f_1,f_2$ are densities, then so is
$f_1 \otimes f_2$.

If $\F$ is a family of densities on $\R$, then by $\F^{\otimes d}$ ($d$-th tensor power of $\F$) we denote
the family of densities on $\R^d$ given by
$$
\F^{\otimes k}=\{f_1 \otimes \ldots \otimes f_d : f_i \in \F\}.
$$




\subsection{Our case}

We assume that we have a family $\F$ larger then Gaussians on $\R$.

We have
$$
\KL(X,\aff(\S^{\otimes d}),\G^d)=\inf_{m,\v} \KL((v^T\v)^{-1}\v^T(X-m),\S^{\otimes d},\G).
$$
Notation: $x[m,\v]$. By the $i$-th coordinate we denote $x[m,\v]_i$.

Thus 
$$
\KL(X,\aff(\S^{\otimes d}),\G^d)=\inf_{m,\v} \left( \sum_{i=1}^d \mle(X[m,\v]_i,\S)
-\mle(X[m,\v],\G)
\right),
$$
where the minus has the direct formula which can be computed.

%%%%%%%%%%%%%%%%%%%%%%%%%%%%%%%%%%%%%%%%%%
\subsection{First approach: global estimation}

We search for the split $g=f \otimes \N$, where $g$ is a normal density on $\R^{D-d}$,
and $f$ is $d$-dimensional. More precisely, we fix a family $\F$ of densities on $\R^d$, and seek $m,\v$ which maximize the MLE:
$$
X \sim a_*(f \otimes g) \text{ for } f \in \F, g \in \nor(\R^{D-d}), a=a_{m,\v} \in \aff(\R^D). 
$$

In the case when $\F$ is one dimensional, the above can be written as:
$$
X \sim \det W \cdot f_1(w_1 \circ (x-m)) \cdot \ldots \cdot f_d(w_d \circ (x-m))
\cdot g_{d+1}(w_{d+1} \circ (x-m)) \cdot \ldots \cdot g_D(w_D \circ (x-m))
$$
where $W=[w_1,\ldots,w_D]=??(V^{-1})^T$.

%%%%%%%%%%%%%%%%%%%%%%%%%%%%%%%%%%%%%%%%%%
\subsection{Number of coordinates}

poszukac - oni cos pisza o ilosci wspolrzednych



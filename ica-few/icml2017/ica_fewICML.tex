%%%%%%%%%%%%%%%%%%%%%%%%%%%%%%%%%%%%%%%%%%%%%%%%%%%%%%%%%%%%%%%%%%
%%%%%%%% ICML 2016 EXAMPLE LATEX SUBMISSION FILE %%%%%%%%%%%%%%%%%
%%%%%%%%%%%%%%%%%%%%%%%%%%%%%%%%%%%%%%%%%%%%%%%%%%%%%%%%%%%%%%%%%%

% Use the following line _only_ if you're still using LaTeX 2.09.
%\documentstyle[icml2016,epsf,natbib]{article}
% If you rely on Latex2e packages, like most moden people use this:
\documentclass{article}

%\usepackage[margin=1in]{geometry}
\usepackage{amsmath,amsfonts,amssymb,amsthm,dsfont}
% use Times
\usepackage{times}
% For figures
\usepackage{graphicx} % more modern
%\usepackage{epsfig} % less modern
\usepackage{subfigure} 

% For citations
%\usepackage{natbib}

% For algorithms
\usepackage{algorithm}
\usepackage{algorithmic}

\usepackage{epstopdf}
% As of 2011, we use the hyperref package to produce hyperlinks in the
% resulting PDF.  If this breaks your system, please commend out the
% following usepackage line and replace \usepackage{icml2016} with
% \usepackage[nohyperref]{icml2016} above.
\usepackage{hyperref}

% Packages hyperref and algorithmic misbehave sometimes.  We can fix
% this with the following command.
\newcommand{\theHalgorithm}{\arabic{algorithm}}

\usepackage{icml2017} 

\icmltitlerunning{Density based ICA for non-square mixing matrix}

% Employ the following version of the ``usepackage'' statement for
% submitting the draft version of the paper for review.  This will set
% the note in the first column to ``Under review.  Do not distribute.''
%\usepackage{icml2016} 

\usepackage{MnSymbol}
%\usepackage{stmaryrd}
\newcommand\de[1]{\lsem #1 \rsem}

\def\N{\mathbb{N}}
\def\Z{\mathbb{Z}}
\def\R{\mathbb{R}}
\def\C{\mathcal{C}}

\def\e{\varepsilon}
\def\d{\delta}
%\def\w{\omega}
\def\w{\mathrm{w}}
\def\V{\mathrm{V}}
\def\v{\mathrm{v}}
\def\x{\mathrm{x}}
\def\m{\mathrm{m}}
\def\z{\mathrm{z}}
\def\s{s}%{{s\I}}

\def\F{\mathcal{F}}
\def\G{\mathcal{G}}
\def\A{\mathcal{A}}
\def\M{\mathcal{M}}
\def\SN{\mathcal{SN}}

\def\1{\mathds{1}}
\def\X{\mathbf{X}}
\def\S{\mathbf{S}}

\def\KL{\mathrm{KL}}

\def\for{\mbox{  for }}
\def\mle{\mathrm{mle}}
\def\diag{\mathrm{diag}}
\def\cov{\mathrm{Cov}}
\def\mean{\mathrm{mean}}
\def\card{\mathrm{card}}
\def\aff{\mathrm{aff}}
\def\span{\mathrm{span}}
\def\nor{\mathcal{N}}
\DeclareMathOperator*{\argmin}{argmin}
\DeclareMathOperator*{\argmax}{argmax}
\def\ICA{ICA$_{\mathcal{S\!N}}$}
\def\SN{\mbox{$\mathcal{S\!N}$}}


\newtheorem{observation}{Observation}[section]
\newtheorem{theorem}{Theorem}[section]
\newtheorem{proposition}{Proposition}[section]
\newtheorem{lemma}{Lemma}[section]
\newtheorem{corollary}{Corollary}[section]

\theoremstyle{definition}
\newtheorem{definition}{Definition}[section]
\newtheorem{remark}{Remark}[section]
\newtheorem{example}{Example}[section]
\newtheorem{problem}{Problem}[section]


% Employ this version of the ``usepackage'' statement after the paper has
% been accepted, when creating the final version.  This will set the
% note in the first column to ``Proceedings of the...''
%\usepackage[accepted]{icml2016}


% The \icmltitle you define below is probably too long as a header.
% Therefore, a short form for the running title is supplied here:
%\icmltitlerunning{ICA - few}

\newcommand\jacek[1]{{\color{red}\sc #1}}

\begin{document} 

\twocolumn[
\icmltitle{Density based ICA for non-square mixing matrix}

% It is OKAY to include author information, even for blind
% submissions: the style file will automatically remove it for you
% unless you've provided the [accepted] option to the icml2017
% package.

% list of affiliations. the first argument should be a (short)
% identifier you will use later to specify author affiliations
% Academic affiliations should list Department, University, City, Region, Country
% Industry affiliations should list Company, City, Region, Country

% you can specify symbols, otherwise they are numbered in order
% ideally, you should not use this facility. affiliations will be numbered
% in order of appearance and this is the preferred way.
%\icmlsetsymbol{equal}{*}

\begin{icmlauthorlist}
\icmlauthor{Przemyslaw Spurek}{goo}
\icmlauthor{Jacek Tabor}{goo}
\icmlauthor{Przemys\l{}aw Rola}{goo}
\end{icmlauthorlist}

\icmlaffiliation{goo}{Jagiellonian University, Krak\'ow, Poland}

\icmlcorrespondingauthor{Przemyslaw Spurek}{przemyslaw.spurek@uj.edu.pl}

% You may provide any keywords that you 
% find helpful for describing your paper; these are used to populate 
% the "keywords" metadata in the PDF but will not be shown in the document
\icmlkeywords{boring formatting information, machine learning, ICML}

\vskip 0.3in
]

\begin{abstract} 
In its basic form Independent Component Analysis (ICA) aims to find an invertible linear transformation of the data so that the resulting data has independent components. In the case when number of sources is less than the number of sensors, the task has an additional step which consists of filtering our the possible noise components. In such situation we are looking for so-called non-square mixing matrix.

Due to computational constraints, principal component analysis is often used for dimension reduction prior to ICA (PCA+ICA). However, such approach often removes important information.
In this paper we present a density based method \ICA{} based 
on split-gaussians which is dedicated for determining non-square mixing matrix in ICA maximum likelihood framework.
\end{abstract} 



%%%%%%%%%%%%%%%%%%%%%%%%%%%%%%%%
ICA is similar in many aspects to principal component analysis (PCA). In PCA we look for an orthonormal change of basis so that the components are not
linearly dependent (uncorrelated).
ICA can be described as a search for the optimal basis (coordinate system) in which the components are independent. Let us now, for the readers convenience, describe how the 
ICA works. The data are represented by the random vector $\x$
 and the components as the random vector~$s$.  Our aim is to transform the observed data $\x$ into maximally independent components $s$ with respect to some measure  
of independence. Here we use a linear static transformation $W$, called the {\em transformation matrix}, combined with the formula 
$$
s = W \x.
$$

Popular ICA methodology does not directly attempt to find components that are independent but rather components that are as non-Gaussian as possible.
This follows from the fact that one of the theoretical foundations of ICA is given by the dual view at the Central Limit Theorem \cite{hyvarinen2000independent}, which states that the distribution of the sum (average or linear combination) of $N$ independent random variables approaches Gaussian as  $N\rightarrow \infty$. Obviously if all source variables are Gaussian, the ICA method will not work. 

There exists many different approaches to ICA which uses negentropy \cite{hyvarinen2000independent}, cumulant-based methods \cite{cardoso1993blind,virta2015joint}, maximum likelihood methods \cite{chen2006efficient,samworth2012independent} and methods that directly minimize a measure of dependence \cite{stogbauer2004least,matteson2016independent}.

In many application the number of sources is unknown and may be less than the number of sensors. In such situation we are looking for so-called non-square mixing matrix.
In practice, PCA is applied to the observations prior to classic ICA (PCA+ICA) to meet the assumption of square mixing and to reduce computational
costs \cite{hyvarinen2004independent}. PCA+ICA is commonly used to identify brain networks
in functional magnetic resonance imaging (fMRI) \cite{beckmann2012modelling,green2002pca} and hyperspectral unmixing \cite{wang2015abundance,caiafa2008blind}.

The problem in such approach is that interesting independent components (ICs) could be mixed in several principal components that are discarded and then these ICs cannot be recovered.

In the paper we present two methods dedicated to a maximum-likelihood framework. In the firs case we are looking directly $d \leq D$ independent component by maximization of likelihood function. The second method work in full dimensional space by  estimating density congaing $d$ non-gaussian components (independent ones) and $D-d$ gaussian ones which model a noise.  

[!!!Opisac w miare dokladnie nasze podejscie!!!]



%%%%%%%%%%%%%%%%%%%%%%%%%%%%%%%%

%%%%%%%%%%%%%%%%%%%%%%%%%%%%%%%%
%%%%%%%%%%%%%%%%%%%%%%%%%%%%%%%%
\section{Related works}
\label{related_works}
%%%%%%%%%%%%%%%%%%%%%%%%%%%%%%%%

In literature exist a few \textbf{jesli masz na mysli malo, to few. a few to duzo} approaches dedicated for non-square ICA problem.
Most of such method are dedicated for \textbf{to zamiast for} individual methods \textbf{dwa razy method w tym zdaniu, nie wiadomo o co chodzi}.
Attias and Schreiner \cite{attias1997blind} derived a likelihood based algorithm for separation of general sequences with a frequency domain implementation.
Belouchrani and Cardoso \cite{belouchrani1995maximum} presented a general likelihood approach allowing
for additive noise and for non-square mixing matrices. They applied the
method to separation of sources taking discrete values\textbf{przecinek} and estimated the
mixing matrix using an Expectation-Maximization (EM) approach with both a deterministic and a stochastic formulation. In \cite{moulines1997maximum} \textbf{the} authors used the EM approach
for separation of autocorrelated sequences in presence of noise and explored
a family of flexible source signal priors based on Gaussian Mixtures. 

The assumption is that \textbf{usun is that} of square mixing is mostly unrealistic in  the case of EEG ant FMRI \textbf{przecinek} where the number of sources is less than the number of electrodes \cite{beckmann2004probabilistic,samarov2004nonparametric,shi2017investigating}. Therefore, many of \textbf{usun of} algorithms dedicated for \textbf{to zamiast for} this task use a probabilistic ICA \cite{tipping1999probabilistic}.
The noisy ICA model can be approximated using a variant of PCA+ICA \cite{beckmann2004probabilistic}, where probabilistic PCA is used to estimate the number of components and achieve dimension reduction \textbf{moze po prostu reduce dimension} \cite{tipping1999probabilistic}.
In \cite{allassonniere2012stochastic} \textbf{the} authors developed stochastic EM algorithms to estimate the noisy model \textbf{przecinek} and proposed parametric methods.

Other methods exploring non-Gaussian structure in multivariate data include non-Gaussian component analysis (NGCA) and projection pursuit \cite{blanchard2006search,kawanabe2007new}.
NGCA is a more general case of linear non-Gaussian component analysis (LNGCA) \cite{risk2015likelihood}
that allows non-linear dependence between the non-Gaussian components.

In the paper  \cite{miettinen2014deflation} \textbf{the} authors propose a novel \textbf{wiadomo ze novel, zamiast a novel po prostu an} adaptive twostage \textbf{two-stage?} deflation-based FastICA algorithm \textbf{przecinek} that allows one to use different nonlinearities for different components \textbf{przecinek} and optimizes the order in which the components are extracted.








%%%%%%%%%%%%%%%%%%%%%%%%%%%%%%%%
  
 %%%%%%%%%%%%%%%%%%%%%%%%%%%%%%%%
%%%%%%%%%%%%%%%%%%%%%%%%%%%%%%%%
\section{Theoretical foundations of \ICA{}}
\label{p}

In this section we present \textbf{the} theoretical foundations of the method.
We begin with the statement of the problem, next we focus our attention on the presentation of \textbf{zamiast tego wszystkiego, next we present} the class of densities we discuss \textbf{used zamiast we discuss}. Last \textbf{przecinek} we show \textbf{compute zamiast show} the gradient of the method \textbf{przecinek} which is needed in the optimization procedure.


\subsection{Statement of the problem}

Since this general \textbf{the zamiast this general} idea of the search for ICA \textbf{for independent components zamiast for ICA} with the use of
maximum likelihood is essential in our further considerations, for the convenience of the reader we first describe it briefly.
Assume that the random vector $\X$ in $\R^D$ has the density function $F(\x)$.
Suppose that the components of $\X$ are not independent, but that
we know (or suspect) that \textbf{moze usunac we know or suspect that?} there is a basis $B$ (we put $W^T=B^{-1}$) such that in that base the
components of $\X$ become independent. Observe \textbf{przecinek} that where \textbf{then zamiast where} $\w_i^T\x$ is the $i$-th coefficient of $\x$ in the basis $B$ ($\w_i$ denotes the $i$-th column of $W$), and therefore there exist densities 
$f_1,\ldots,f_D$ such that
\begin{equation} \label{eq:gen}
F(\x)=\det(W) \cdot f_1(\w_1^T\x) \cdot \ldots \cdot f_d(\w_d^T\x).
\end{equation}
Given $W$ and densities $(f_i)_{i=1}^D$ we introduce notation to represent
RHS \textbf{we denote the right-hand side} of the above equation \textbf{as follows}:
$$
F_W(f_1,\ldots,f_D)(\x)=\det(W) \cdot f_1(\w_1^T\x) \cdot \ldots \cdot f_d(\w_d^T\x).
$$
Thus we may \textbf{Let us now zamiast Thus we may} state the density based formulation of ICA in the case we have only a sample
$X$ from random vector $\X$.

\medskip

\noindent \textbf{THE?} GENERAL ICA PROBLEM (\textbf{the} maximum likelihood formulation). \\{\em Find densities $f_i$ and matrix $W$, so that $F$ given by
\eqref{eq:gen} optimally fits the data $X=(\x_i)$ with respect to the likelihood, that is that the value
$$
\sum_i \log F_W(f_1,\ldots,f_D)(\x_i)
$$
is maximized.
}

\medskip

Since the search over the space of all densities is not feasible, and could lead to overfitting, we naturally have to reduce to a subclass of all densities on $\R$ parametrized by a finite amount of parameters. Clearly, since 
ICA does not work if the data are gaussian, we have to choose a family $\F$ of densities which is distant from Gaussian ones. 

\medskip

\noindent ICA FOR DENSITY CLASS $\F$.\\ {\em Find \textbf{a} matrix $W$ and
densities
$
f_1,\ldots,f_D \in \F, 
$
such that the value of 
$$
\sum_i \log F_W(f_1,\ldots,f_D)(\x_i)
$$
is maximized.
}

\medskip

Similarly to \cite{ICA2017pattern} as a class \textbf{usun a class} $\F$ we are going to
take the class of split-gaussians, as 
as it is easy to deal with (small number of parameters) and is resistant to outliers\footnote{The reason is that split gaussians \textbf{moze split gaussian distributions}, instead at fitting the distribution with respect to heavy tails, fits the asymmetry of the data. \textbf{troche to zdanie ciezkie. Moze po prostu split gaussians are fitting well to asymmetrical data?}}.

As mentioned in the introduction, we assume that components which we would like to filter-out, \textbf{tu bez przecinka} are coming from a gaussian noise, and the \textbf{then zamiast the} aim it to fit the first $d$-components from a larger class of densities, while the rest from the gaussians $\nor$.\textbf{Poprzednie zdanie jest troche niezrozumiale, przepisz prosze (moze rozbij na dwa).} Thus our final problem can be stated as follows.

\medskip

\noindent ICA FOR DENSITY CLASS $\F$ WITH $d$ SOURCES. \\{\em Find \textbf{a} matrix $W$, densities 
$
f_1,\ldots,f_d \in \F \text{ and normal densities } f_{d+1},\ldots,f_D \in \nor,
$
so that the value of 
$$
\sum_i \ln F_W(f_1,\ldots,f_D)(\x_i)
$$
is maximized.
}

\medskip

Observe that the solution to the above problem is linearly invariant, that is if
$W$ is optimal for $X$ an $A$ is linear, then $W_A$ is optimal for $AX$,
where $W_A=(A^{-1})^TW$.

The continuous version of the condition we maximize in the case we know the density $f$
of the random variable $\X$ limits to
$$
\begin{array}{l}
\int \ln F_W(f_1,\ldots,f_D)(\x)f(\x) d\x %\\[1ex]
=-H(f,F_W(f_1,\ldots,f_D)),
\end{array}
$$
where the cross entropy $H(f,g)$ is given by the sum of entropy $H(f)$
and Kullback-Leibler divergence $D_{KL}(f,g)$. Thus the continuous version of the ICA problem
with $d$ sources reduces to the minimization of 
$$
D_{KL}(f,F_W(f_1,\ldots,f_D))
$$
over all matrices $W$ and densities $f_1,\ldots,f_D \in \F$. Since for fixed $f$ Kullback-Leibler divergence is minimized for $g=f$, we arrive at the following result, which says that in the ideal case 
by the discussed approach we restore the unmixing matrix if it exists.

\begin{theorem}
  Let $F$ be a density such that there exist \textbf{a} matrix $\overline W$ and densities
$$
\hat f_1,\ldots,\hat f_d \in \F \text{ and } \hat f_{d+1},\ldots,\hat f_D \in \nor
$$
such that 
$$
F=F_{\overline W}(\hat f_1,\ldots,\hat f_D).
$$
Then
$$
\begin{array}{l}
\overline W,\hat f_1,\ldots,\hat f_D %\\[1ex]
\begin{array}{l}
=\argmin \{F_{W}(f_1,\ldots,f_D): %\\[1ex] 
%\phantom{=\argmin \{}
W, f_1,\ldots,\bar f_d \in \F,f_{d+1},\ldots,f_D \in \nor
\}.
\end{array}
\end{array}
$$
\end{theorem}

\subsection{Split normal distribution}


In this section we discuss the class $\F$ we will use in our final algorithm of \ICA{}. The density of \SN{}, the one-dimensional split normal distribution \cite{villani2006multivariate}, is given by the formula
$$
\SN(x;m,\sigma^2,\tau^2) = \left\{ \begin{array}{l}
c \cdot \exp[-\frac{1}{2\sigma^2}(x-m)^2], \textrm{$x\leq m$}\\
c \cdot \exp[-\frac{1}{2\tau^2\sigma^2}(x-m)^2], \textrm{$x>m$}\\
\end{array} \right. \!\!\!\!,
$$
where $c=\sqrt{\frac{2}{\pi}}\sigma^{-1}(1+\tau)^{-1}$. 


As we see \textbf{przecinek} the split normal distribution comes from merging two opposite halves of two normal distributions in their common mode. The main advantage of \textbf{in using zamiast of} split normal distributions over normal one \textbf{moze over regular ones?} is that it \textbf{they zamiast it} allows \textbf{allow zamiast allows} data asymmetry. In 1982 John \cite{john1982three} showed \textbf{Moze It was shown by John.. i bez pisania roku} that the likelihood function can be expressed in a  form in which the scale parameters $\sigma$ and $\tau$ are an explicit function of the location parameter $m$.
In the case when $\F=\SN$
the density class considered in the previous subsection is given in the
explicit form by the following observation.

\begin{observation}\label{def:GSN}
  \textbf{Czemu Observation 31? Przenumeruj to jakos} A density of the multivariate split normal $d$ and normal $D-d$ distribution is given by
$$
\begin{array}{l}
\SN_{d}\nor_{D-d}(\x; \m,W, \sigma^2,\tau^2)=\\[6pt]
\det(W) \prod \limits_{j=1}^{d} \SN(\w_j^T(\x-\m);0,\sigma_j^2,\tau_j^2)%\cdot\\[1ex]
\cdot \prod \limits_{j=d+1}^{D} \nor(\w_j^T(\x-\m);0,\sigma_j^2),
\end{array}
$$
where $\w_{j}$ is the $j$-th column of non-singular matrix $W$, $\m = (m_1, \ldots, m_d)^T$, $\sigma = (\sigma_{1},\ldots,\sigma_{d})$ and $\tau=(\tau_{1},\ldots,\tau_{D-d})$.
\end{observation}

Observe that the above density probability function has mode in $\m$.
As a consequence of result of John \cite{john1982three} we can maximize the likelihood of the above function on data $X$ with respect to $\sigma$ and $\tau$.

\begin{theorem}\label{the:min}
Let $\x_1,\ldots,\x_n$ be given, and let $\m \in \R^D$ and matrix $W$
be fixed.  
Then the likelihood maximized w.r.t. $\sigma$ and $\tau$ is
\begin{equation}\label{eq:1}
\begin{array}{l}
 \hat{L}(X;\m,W) =   \frac{ 2^{(d-D/2)n} n^{dn/2} }{(\pi e)^{Dn/2}} %\cdot \\[1ex]
 \bigg( \frac{1}{|\det(W)|^{\frac{2}{3}}} \prod\limits_{j=1}^{d} g_{j}(\m,W) \bigg)^{-3n/2} 
\bigg( \prod\limits_{j=d+1}^{D} \frac{(s_1+s_2)}{n} \bigg)^{-n/2},
\end{array}
\end{equation}
where
$$
\begin{array}{c}
{g}_{j}(\m,W) = {s}_{1j}^{1/3} + {s}_{2j}^{1/3},
\\[1ex]
{s}_{1j}= \! \sum\limits_{i \in I_j}[ \w_{j}^T (\x_i-\m)]^2,  {I}_j=\{ i  \colon \w_{j}^T (\x_i-\m) \leq 0 \},
\\[1ex]
{s}_{2j}= \! \sum\limits_{i \in I_j^c}[ \w_{j}^T (\x_i-\m)]^2, {I}_j^c=\{ i \colon  \w_{j}^T (\x_i-\m) > 0 \},
\end{array}
$$
and the maximum likelihood estimators of $\sigma_{j}^2$ and $\tau_{j}$ are
\begin{equation}\label{eq:est}
\begin{array}{l}
\hat \tau_{j}(\m,W)=\left(\frac{s_{2j}}{s_{1j}}\right)^{1/3}, \qquad 1 \leq j \leq d\\[6px]
\hat \sigma_j^2(\m,W) = \left\{ \begin{array}{l}
\tfrac{1}{n} s_{1j}^{2/3} g_{j}(\m,W), \; 1 \leq j \leq d\\
\tfrac{1}{n} (s_{1j}+s_{2j}), \qquad d < j \leq D\\
\end{array} \right. \!\!\!\!.
\end{array}
\end{equation}
\end{theorem}
%\comment{Przemek R.- task 3. sprawdzic dowod Theorem \ref{the:min}}

\begin{proof}
See Section \ref{a1} (Appendix A).
\end{proof}

Thanks to the above theorem we can reduce the search for the maximum of the log-likelihood function for two parameters $\m \in \R^d$ and $W \in \M(\R^d)$.

\begin{equation}\label{equ:ll}
{l}(X;\m,W) = \frac{1}{|\det(W)|^{\frac{2}{3}}} \prod_{j=1}^{d} {g}_{j}(\m,W) \prod_{j=d+1}^{D} (s_{1j} + s_{2j})^{\frac{1}{3}}
\end{equation}
where $w_{j}$ stands for the $j$-th column of matrix $W$. 
Consequently, maximization of likelihood function is equivalent to minimization of  $\ln l$.

\begin{corollary}\label{c2}
Let $X \subset \R^d$, $\m \in \R^d$, $W \in \M(\R^d)$ be given, then
$$
\argmax_{\m,W} \hat{L}(X;\m,W) =  \argmin_{\m,W} \ln {l}(X;\m,W).
$$
\end{corollary}

%\subsection{Optimization}

To minimize $\ln l$ with the use classical gradient descent method \textbf{moze standard gradient methods? We wtyczce uzywamy BFGSa na przyklad to jakas wariacja gradient descentu},  we need
the formula for $\nabla \ln l$ (\textbf{the} gradient of the cost function). 

%Now we are ready to calculate gradient of our cost function.

%\begin{theorem}\label{ther:grad}
%Let $X \subset \R^d$, $\m = (\m_1, \ldots, \m_d)^T \in \R^d$, $W = (\w_{ij})_{1 \leq i,j \leq d}$ non-singular be given. 
%Then
%$\nabla_{\m}  \ln {l}(X;\m,W) = \left(  \frac{\partial \ln {l}(X;\m,W)}{\partial \m_1}, \ldots, \frac{\partial \ln {l}(X;\m,W)}{\partial \m_d} \right)^T$,
%where
%$$
%\begin{array}{l}
%\frac{\partial \ln {l}(X;\m,W)}{\partial \m_k} =
%\end{array}
%$$
%Moreover,
%$
%\nabla_{W} \ln {l}(X;\m,W) = \left[ \frac{\partial \ln \tilde{l}(X;\m,W)}{\partial \w_{pk}}  \right]_{1 \leq p,k \leq d},
%$
%where
%$$
%\begin{array}{l}
%\frac{\partial \ln \tilde{l}(X;\m,W)}{\partial \w_{pk}}  = .
%\end{array}
%$$
%and
%$$
%\begin{array}{c}
%%{g}_{j}(\m,W) = {s}_{1j}^{1/3} + {s}_{2j}^{1/3},
%%\\[1ex]
%{s}_{1j}= \! \sum\limits_{i \in I_j}[ \w_{j}^T (\x_i-\m)]^2, {I}_j=\{ i  \colon \w_{j}^T (\x_i-\m) \leq 0 \},
%\\[1ex]
%{s}_{2j}= \! \sum\limits_{i \in I_j^c}[ \w_{j}^T (\x_i-\m)]^2,  {I}_j^c=\{ i  \colon  \w_{j}^T (\x_i-\m) > 0 \}.
%\end{array}
%$$
%\end{theorem}

\begin{theorem}\label{ther:grad}
Let $X \subset \R^d$, $\m = (\m_1, \ldots, \m_d)^T \in \R^d$, $W = (\w_{ij})_{1 \leq i,j \leq d}$ non-singular be given. 
Then
$\nabla_{\m}  \ln {l}(X;\m,W) = \left(  \frac{\partial \ln {l}(X;\m,W)}{\partial \m_1}, \ldots, \frac{\partial \ln {l}(X;\m,W)}{\partial \m_d} \right)^T$,
where
$$
\begin{array}{l}
\frac{\partial \ln {l(X;\m,W)}}{\partial \m_k} =\sum\limits_{j=1}^d \frac{-2}{3({s}_{1j}^{\frac{1}{3}} + {s}_{2j}^{\frac{1}{3}})} \bigg(
\frac{1}{{s}_{1j}^{\frac{2}{3}}} \sum\limits_{i \in I_j} \w_j^T (\x_i - \m)  \w_{jk} +% \\[6pt]
\frac{1}{{s}_{2j}^{\frac{2}{3}}} \sum\limits_{i \in I_j^c} \w_j^T (\x_i - \m) \w_{jk}
\bigg)+ \\[6pt]
\sum\limits_{j=d+1}^D \frac{-2}{3(s_{1j}+s_{2j})} \cdot %\\[6pt]
\bigg(
 \sum\limits_{i \in I_j} \w_j^T (\x_i - \m)  \w_{jk} + %\\[6pt]
 \sum\limits_{i \in I_j^c} \w_j^T (\x_i - \m) \w_{jk}
\bigg).
\end{array}
$$
Moreover,
$
\nabla_{W} \ln {l}(X;\m,W) = \left[ \frac{\partial \ln \tilde{l}(X;\m,W)}{\partial \w_{pk}}  \right]_{1 \leq p,k \leq d},
$
where
$$
\begin{array}{l}
\frac{\partial \ln {l(X;\m,W)}}{\partial \w_{pk}} = -\frac{2}{3} (\w^{-1})^T_{pk} +\\[6pt]
 \frac{2}{3({s}_{1p}^{\frac{1}{3}} +{s}_{2p}^{\frac{1}{3}})} 
 \bigg(
{s}_{1p}^{-\frac{2}{3}} \sum\limits_{ i \in {I}_p} \w^T_p (\x_i - \m) (\x_{ik} - \m_k)
+ {s}_{2p}^{-\frac{2}{3}} \sum\limits_{ i \in {I}_p^c} \w^T_p (\x_i - \m) (\x_{ik} - \m_k) \bigg)+ \\[6pt]
\frac{2}{ 3(s_{1p}+s_{2p}) } \bigg( 
\sum\limits_{ i \in {I}_p} \w^T_p (\x_i - \m) (\x_{ik} - \m_k) + \sum\limits_{ i \in {I}_p^c} \w^T_p (\x_i - \m) (\x_{ik} - \m_k) \bigg),
\end{array}
$$
and
$$
\begin{array}{c}
%{g}_{j}(\m,W) = {s}_{1j}^{1/3} + {s}_{2j}^{1/3},
%\\[1ex]
{s}_{1j}= \! \sum\limits_{i \in I_j}[ \w_{j}^T (\x_i-\m)]^2, {I}_j=\{ 1 \leq i \leq n \colon \w_{j}^T (\x_i-\m) \leq 0 \},
\\[1ex]
{s}_{2j}= \! \sum\limits_{i \in I_j^c}[ \w_{j}^T (\x_i-\m)]^2,  {I}_j^c=\{ 1 \leq i \leq n \colon  \w_{j}^T (\x_i-\m) > 0 \}.
\end{array}
$$
\end{theorem}

\begin{proof}
See Section \ref{a2} (Appendix B).
\end{proof}


Thanks to the above Theorem we are able to use in our experiments the gradient descent for finding the minimum of our cost function.



%%%%%%%%%%%%%%%%%%%%%%%%%%%%%%%%

%%%%%%%%%%%%%%%%%%%%%%%%%%%%%%%%%%%%%%%%%%%%%%%%%%%%%%%%%%%%%%%%
%%%%%%%%%%%%%%%%%%%%%%%%%%%%%%%%
\section{Experiments}
\label{experiments}


\begin{table*}[t]
\caption{Tucker's congruence coefficients for reconstruction of two images.}
\label{tab:congru_img_1}
\vskip 0.15in
\begin{center}
\begin{small}
\begin{sc}
\resizebox{\textwidth}{!}{
\begin{tabular}{ c c c c c c c c c  c  c  c  c }
\hline
%\abovespace\belowspace
 & \ICA  & FastICA & FastICA & FastICA & Infomax & Infomax & Infomax & JADE & PearsonICA & ProDenICA & FixNA \\ 
  &  &  logcosh & exp & kurtosis & tanh & tangent & logistic &  & & &  \\
\hline
%\abovespace
4.1.01 & 0.99984  &  0.59585  &  0.59967  &  0.58722  &  0.59596  &  0.58965  &  0.59589  &  0.58333  &  0.46302  &  0.83162  &  0.99948  \\
4.1.02 & 1  &  0.93572  &  0.93315  &  0.94044  &  0.93564  &  0.93925  &  0.93569  &  0.94215  &  0.92294  &  0.98521  &  0.99325  \\ \hline
4.1.06 & 0.9959  &  0.61189  &  0.61636  &  0.59603  &  0.61599  &  0.60059  &  0.69578  &  0.5629  &  0.5971  &  0.99852  &  0.99937  \\
4.1.03 & 0.8676  &  0.79174  &  0.78969  &  0.7982  &  0.78986  &  0.79647  &  0.72715  &  0.80803  &  0.79781  &  0.87541  &  0.99811  \\ \hline
4.2.04 & 0.99955  &  0.60087  &  0.6007  &  0.60071  &  0.60088  &  0.60094  &  0.60087  &  0.59933  &  0.57862  &  0.98594  &  0.88695  \\
5.2.10 & 0.99974  &  0.94413  &  0.9406  &  0.95526  &  0.94426  &  0.94951  &  0.94395  &  0.96464  &  0.98256  &  0.97757  &  0.88972  \\ \hline
4.2.02 & 0.99679  &  0.59655  &  0.59712  &  0.59554  &  0.59658  &  0.59538  &  0.59656  &  0.59434  &  0.59743  &  0.99237  &  0.96088  \\
5.2.08 & 0.99021  &  0.97131  &  0.96951  &  0.97395  &  0.97122  &  0.97433  &  0.97128  &  0.97647  &  0.96843  &  0.98366  &  0.94757  \\ \hline
boat.512 & 0.99732  &  0.58702  &  0.58689  &  0.58784  &  0.58692  &  0.58748  &  0.58623  &  0.5865  &  0.58728  &  0.99835  &  0.93781  \\
5.3.01 & 0.80906  &  0.98201  &  0.98196  &  0.98213  &  0.98197  &  0.98213  &  0.98163  &  0.98177  &  0.98209  &  0.97381  &  0.97218  \\ \hline
elaine.512 & 0.96643  &  0.58926  &  0.58919  &  0.59017  &  0.58922  &  0.58987  &  0.58935  &  0.58929  &  0.58901  &  0.65476  &  0.71303  \\
4.2.03 & 0.99839  &  0.96629  &  0.96628  &  0.96641  &  0.96628  &  0.96638  &  0.96631  &  0.9663  &  0.96624  &  0.99326  &  0.79288  \\ \hline
119082 & 0.99995  &  0.59432  &  0.59502  &  0.59347  &  0.59391  &  0.59374  &  0.59492  &  0.58197  &  0.58604  &  0.99233  &  0.76561  \\
157055 & 0.99804  &  0.94331  &  0.94273  &  0.94397  &  0.94363  &  0.94377  &  0.94281  &  0.94928  &  0.94804  &  0.93969  &  0.73722  \\ \hline
42049 & 0.9999  &  0.63238  &  0.61046  &  0.62306  &  0.63239  &  0.62743  &  0.6324  &  0.61757  &  0.5872  &  0.93106  &  0.88476  \\
220075 & 0.99845  &  0.79121  &  0.70406  &  0.91538  &  0.79133  &  0.90155  &  0.79148  &  0.92798  &  0.96143  &  0.9678  &  0.93793  \\ \hline
43074 & 0.98658  &  0.58577  &  0.58375  &  0.59414  &  0.58555  &  0.59142  &  0.58612  &  0.59278  &  0.55845  &  0.57092  &  0.78144  \\
295087 & 0.99757  &  0.97911  &  0.98011  &  0.97119  &  0.97923  &  0.97465  &  0.97891  &  0.97306  &  0.97978  &  0.99416  &  0.80112  \\ \hline
38092 & 0.95367  &  0.58677  &  0.58678  &  0.58699  &  0.58677  &  0.58692  &  0.58675  &  0.587  &  0.58572  &  0.38511  &  0.76369  \\
167062 & 0.99999  &  0.9916  &  0.9916  &  0.99139  &  0.9916  &  0.99148  &  0.99161  &  0.99137  &  0.99148  &  0.99881  &  0.74187  \\ \hline
\end{tabular}
}
\end{sc}
\end{small}
\end{center}
\vskip -0.1in
\end{table*}




To compare \ICA{} to other state-of-the-art approaches we use 
Tucker's congruence coefficient \cite{lorenzo2006tucker} which values range between $-1$ and $+1$. It can be used to study the similarity of extracted factors across different samples. Generally, a congruence coefficient of $0.9$ indicates a high degree of factor similarity, while a coefficient of $0.95$ or higher indicates that the factors are virtually identical. 

We evaluate our method in the context of 2D and hyperspectral images. 
For comparison we use R package {\tt ica} \cite{ica}, {\tt PearsonICA} \cite{pearsonica}, {\tt ProDenICA} \cite{prodenica}, {\tt tsBSS} \cite{tsBSS}.
The most popular method used in practice is FastICA \cite{hyvarinen1999fast,helwig2013critique} algorithm, which uses negentropy. In this context we can use three different functions to estimate neg-entropy:
logcosh, exp and kurtosis.
We also compare our method with algorithm using Information-Maximization (Infomax) approach \cite{bell1995information}. Similarly to FastICA we consider three possible non-linear functions: hyperbolic tangent, logistic and extended Infomax.
%We also consider algorithm which uses Joint Approximate Diagonalization of Eigenmatrices (JADE) proposed by Cardoso and Souloumiac's \cite{cardoso1993blind,helwig2013critique}.
%
%One of the most popular ICA methods dedicated for skew data is PearsonICA \cite{karvanen2000pearson,karvanen2002blind}, which minimizes mutual information using a Pearson \cite{stuart1968advanced} system-based parametric model. Another model we consider is ProDenICA \cite{bach2002kernel,hastie2009elements}, which is based not on a
%single nonlinear function, but on an entire function space of candidate nonlinearities. In particular, the method works with the functions in a reproducing kernel Hilbert space, and make use of the “kernel trick” to search over this space efficiently. 
%We also compare our method with  FixNA \cite{shi2009blind}, method for blind source separation problem.
%




\subsection{Separation of images}

One of the most popular application of ICA is the separation of images. In our experiments we use four images from the USC-SIPI Image Database of size $256 \times 256$ pixels (4.1.01, 4.1.06, 4.1.02, 4.1.03) and eight of size $512 \times 512$ pixels (4.2.04, 4.2.02, boat.512, elaine.512, 5.2.10, 5.2.08, 5.3.01, 4.2.03). We also use 8 images from the Berkeley Segmentation Dataset of size $482 \times 321$ with indexes (\#119082, \#42049, \#43074, \#38092, \#157055, \#220075, \#295087, \#167062). 

We make random pairs of above images and one component with noise (random sample from Gaussian distribution $\nor(0,1)$) and use them as a source signal combined by the mixing matrix $A = \begin{bmatrix} 1 & 1 & 1  \\ 1 & -1 & -1 \\ -1 & 1 & -1  \end{bmatrix} $. Our goal was to reconstruct two original images by using only the knowledge about mixed ones. The visualization of this process we present in Fig. \ref{fig:image_ICA_int}. The results of this experiment are presented in Tab.~\ref{tab:congru_img_1} where we present Tucker's congruence coefficients which shows that almost in all cases \ICA{} obtains best results. This is illustrated in Figure \ref{fig:image_ICA_int}, where we can see that \ICA {} almost perfectly recovered source signal. 
Although this is not surprising as the experiments were in fact conducted in the setting which favored our approach, as we chose the noise to be gaussian, this shows that \ICA{} works as desired and deals well with removing gaussian
components from the data. 



%\begin{landscape}
\begin{figure*}[t!]
% ensure that we have normalsize text
\normalsize
\begin{center}
%\subfigure[The effect of the \ICA \ method.] {\label{fig:image_1}
%  \includegraphics[width=1.2in]{spec/1a} 
%  \includegraphics[width=1.2in]{spec/2a}
%  \includegraphics[width=1.2in]{spec/3a} 
%  \includegraphics[width=1.2in]{spec/4a}
%} \\
%%%%%%%%%%%%%%%%%%%%%%%%%%%%%%%%%%
%%%%%%%%%%%%%%%%%%%%%%%%%%%%%%%%%%
\subfigure[Ground truth layers which contains 4 channels: \#1 Asphalt, \#2 Grass, \#3 Tree and \#4 Roof.] {\label{fig:image_1}
  \includegraphics[width=1.2in]{spec/O_1} 
  \includegraphics[width=1.2in]{spec/O_2}
  \includegraphics[width=1.2in]{spec/O_3} 
  \includegraphics[width=1.2in]{spec/O_4}
} \\
%%%%%%%%%%%%%%%%%%%%%%%%%%%%%%%%%%
%%%%%%%%%%%%%%%%%%%%%%%%%%%%%%%%%%
\subfigure[The effect of the \ICA \ method.] {\label{fig:image_1}
  \includegraphics[width=1.2in]{spec/SG_1} 
  \includegraphics[width=1.2in]{spec/SG_2}
  \includegraphics[width=1.2in]{spec/SG_3} 
  \includegraphics[width=1.2in]{spec/SG_4}
} \\
%%%%%%%%%%%%%%%%%%%%%%%%%%%%%%%%%%
%%%%%%%%%%%%%%%%%%%%%%%%%%%%%%%%%%
\subfigure[The effect of the FastICA (logcosh) method.] {\label{fig:image_1}
  \includegraphics[width=1.2in]{spec/ICA11_1} 
  \includegraphics[width=1.2in]{spec/ICA11_2}
  \includegraphics[width=1.2in]{spec/ICA11_3} 
  \includegraphics[width=1.2in]{spec/ICA11_4}
} \\
%%%%%%%%%%%%%%%%%%%%%%%%%%%%%%%%%%
%%%%%%%%%%%%%%%%%%%%%%%%%%%%%%%%%%
%\subfigure[The effect of the PearsonICA method.] {\label{fig:image_1}
%  \includegraphics[width=1.2in]{spec/ica_41} 
%  \includegraphics[width=1.2in]{spec/ica_42}
%  \includegraphics[width=1.2in]{spec/ica_43} 
%  \includegraphics[width=1.2in]{spec/ica_44}
%} \\
%%%%%%%%%%%%%%%%%%%%%%%%%%%%%%%%%%
%%%%%%%%%%%%%%%%%%%%%%%%%%%%%%%%%%
\subfigure[The effect of the  ProDenICA method.] {\label{fig:image_1}
  \includegraphics[width=1.2in]{spec/ICA5_1} 
  \includegraphics[width=1.2in]{spec/ICA5_2}
  \includegraphics[width=1.2in]{spec/ICA5_3} 
  \includegraphics[width=1.2in]{spec/ICA5_4}
} \\
%%%%%%%%%%%%%%%%%%%%%%%%%%%%%%%%%%
%%%%%%%%%%%%%%%%%%%%%%%%%%%%%%%%%%
\end{center}
\caption{Results of image separation with the uses of various ICA algorithms.}
\label{fig:spec_1}
\end{figure*}


\subsection{Hyperspectral Unmixing}

Independent component analysis has been recently
applied into hyperspectral unmixing 
\cite{wang2015abundance, caiafa2008blind}
as a result of its low
computation time and its ability to perform without prior information.
In this subsection we apply simple example which suggests that our method also can by used for spectral data.

Urban data  \cite{fyzhu2014IJPRSSSNMF,fyzhu2014TIPDgSNMF,fyzhu2014JSTSPRRLbSF} is one of the most widely used hyperspectral data-sets used in the hyperspectral unmixing study. Each image has $307 \times 307$ pixels, each of which corresponds to a $2 \times 2$ m area. In this image, there are 210 wavelengths ranging from 400 nm  to 2500 nm, resulting in a spectral resolution of 10 nm. After the channels 1--4, 76, 87, 101--111, 136--153 and 198--210 are removed (due to dense water vapor and atmospheric effects), there remain 162  channels (this is a common preprocess for hyperspectral unmixing analyses). There is ground truth \cite{fyzhu2014IJPRSSSNMF,fyzhu2014TIPDgSNMF,fyzhu2014JSTSPRRLbSF}, which contains 4 channels: \#1 Asphalt, \#2 Grass, \#3 Tree and \#4 Roof.

A highly mixed area is cut from the original data set in this experiment (similar example was showed in \cite{wang2015abundance}), with the size of $200 \times 150$ pixels. %Fig. \ref{} shows the true-color image. 

In our experiment we compared \ICA{} to other two popular ICA methods --
ProDenICA and FastICA, see Fig. \ref{fig:spec_1}. Observe that \ICA{} and ProDenICA give layers which seem to contain more information then FastICA, as the last component in FastICA contains mainly noise.


%
%\begin{table*}[t]
%\caption{Classification accuracies for naive Bayes and flexible 
%Bayes on various data sets.}
%\label{tab:congru_img_1}
%\vskip 0.15in
%\begin{center}
%\begin{small}
%\begin{sc}
%\resizebox{\textwidth}{!}{
%\begin{tabular}{ c c c c c c c c c  c  c  c  c }
%\hline
%%\abovespace\belowspace
% & \ICA  & FastICA & FastICA & FastICA & Infomax & Infomax & Infomax & JADE & PearsonICA & ProDenICA \\ 
%  &  &  logcosh & exp & kurtosis & tanh & tangent & logistic &  & &  \\
%\hline
%%\abovespace
%
%\hline
%\end{tabular}
%}
%\end{sc}
%\end{small}
%\end{center}
%\vskip -0.1in
%\end{table*}

%\begin{table*}[!t]
%\centering
%\scalebox{0.7}{ 
%\begin{tabular}{ | c | c  | c c c | c c c | c | c | }
%\multicolumn{1}{c}{} & \multicolumn{1}{c}{\ICA}  & \multicolumn{3}{c}{FastICA} & \multicolumn{3}{c}{Infomax}  & \multicolumn{1}{c}{PearsonICA} & \multicolumn{1}{c}{ProDenICA}  \\ 
% &  &  logcosh & exp & kurtosis & tanh & tangent & logistic & &  \\
%\hline
%\#1 Asphalt  &\bf 0.6774 &  0.2859 & 0.2864 & -0.2595 & -0.2972 & -0.2954 &  -0.2972 &  0.20978 & 0.4928  \\
%\#2 Grass  & \bf -0.7784 &  -0.2746 & -0.2605 &  -0.2798 &  -0.2814 & -0.2816 & -0.2814 &  -0.2412 & -0.4323  \\
%\#3 Tree  & \bf 0.7267 & 0.2338 &  0.2717 &  -0.2547 &  0.2441 & 0.2354 &  0.2442 &   0.2482 & -0.5961  \\
%\#4 Roof &\bf 0.6666 &  -0.4256 &  0.4279 &  0.4167 & -0.4244 &  0.4301 & -0.4244 &  0.4193 &  -0.6128  \\
%
%\end{tabular}
%}
%\caption{The Tucker's congruence coefficient measure between reference layers and results of different ICA algorithms in the case of the urban data set.}
%\label{tab:spec}
%\end{table*}


%\begin{figure*}[!t]
%\normalsize
%\begin{center}
%\includegraphics[width=5in]{spec_1}
%\end{center}
%\caption{Congruence distance between layers obtain by different ICA algorithms and the closest reference channel.}
%\label{fig:spec_1}
%\end{figure*}





%%%%%%%%%%%%%%%%%%%%%%%%%%%%%%%%

    
%%%%%%%%%%%%%%%%%%%%%%%%%%%%%%%%
\section{Appendix A}
\label{a1}
%%%%%%%%%%%%%%%%%%%%%%%%%%%%%%%%%%%%%%%%%%%%%%%%%%%%%%%%%%%%%%%%
\begin{proof}[Proof of Theorem \ref{the:min}.]
\textbf{Zrob tak zeby sie proof nie powtarzal 2x w naglowku w tym i innych twierdzeniach w appendixie}
Let $X=\{ \x_1, \ldots, \x_n \}$.
We write 
\begin{equation*}
\z_i= W(\x_i-m), \quad \z_{ij}= \w_j^T(\x_i-m),
\end{equation*}
for observation $i$, where $i=1,\ldots,n$ and coordinates $j=1,\ldots,d$.

Let us consider the likelihood function, i.e. 
$$
\begin{array}{l}
L(X;\m,W,\sigma,\tau) = \prod\limits_{i=1}^{n} SN_{d}N_{D-d}(\x_i; \m,W, \sigma^2,\tau^2) \\[6pt] 
=\prod\limits_{i=1}^{n} | \det(W)|  \prod\limits_{j=1}^{d} SN(  \w_j^T (\x_i - \m) ; 0 , \sigma_j^2, \tau_j^2) \cdot %\\[6pt]
\prod\limits_{j=d+1}^{D} N(  \w_j^T (\x_i - \m) ; 0 , \sigma_j^2) =\\[6pt]
\Big( c_1|\det(W)| \Big)^{n} 
\Big( \prod\limits_{j=1}^{d} \sigma_j(1+\tau_j) \Big)^{-n} %\cdot \\[6pt]
\prod\limits_{i=1}^{n} %\\[6pt]
\prod\limits_{j=1}^{d} \exp \Big[ -\frac{1}{2\sigma_j^2}z_{ij}^2 (\1_{ \{ z_{ij} \leq 0 \} } + %\\[6pt]
\tau_{j}^{-2} \1_{ \{ z_{ij} > 0 \} }) \Big] \\[6pt]
\Big( \prod\limits_{j=d+1}^{D} \sigma_j \Big)^{-n} \prod\limits_{i=1}^{n}\prod\limits_{j=d+1}^{D} \exp \Big[ -\frac{1}{2\sigma_j^2}z_{ij}^2 \Big],
\end{array}
$$
%$$
%\begin{array}{l}
%= \prod\limits_{i=1}^{n} GSN_d(\x_i ; \m,V,\sigma,\tau)
%= \prod\limits_{i=1}^{n} \frac{1}{| \det( V)|}  \prod\limits_{j=1}^{d} SN(  \v^{-1}_j \x_i ; \v^{-1}_j \m , \sigma_j^2, \tau_j^2)=
%\\[1ex]
%\left(\frac{c_1}{|\det(V)|}\right)^{n} \left( \prod\limits_{j=1}^{d} \sigma_j(1+\tau_j) \right)^{-n} \left( \prod\limits_{i=1}^{n} \prod\limits_{j=1}^{d} \exp[-\frac{1}{2\sigma_j^2}z_{ij}^2 (\1_{ \{ z_{ij} \leq 0 \} } + \tau_{j}^{-2} \1_{ \{ z_{ij} > 0 \} })]\right)
%\end{array}
%$$
where 
$
c_1= \left( \sqrt{\tfrac{2}{\pi}} \right)^{d} \cdot \left( \tfrac{1}{ \sqrt{2\pi} } \right)^{D-d}.
$
Now we take the log-likelihood function, i.e. 
$$
\begin{array}{l}
\ln(L(X;\m,W,\sigma,\tau)) = \ln \bigg( \Big( c_1|\det(W)| \Big)^{n} \Big( \prod\limits_{j=1}^{d} \sigma_j(1+\tau_j) \Big)^{-n} \Big( \prod\limits_{j=d+1}^{D} \sigma_j \Big)^{-n}  \bigg) + \\[6pt]
 \sum\limits_{i=1}^{n} \sum\limits_{j=1}^{d} \Big[ -\frac{1}{2\sigma_j^2}z_{ij}^2 (\1_{ \{ z_{ij} \leq 0 \} } + \tau_{j}^{-2} \1_{ \{ z_{ij} > 0 \} })\Big] +  %\\[6pt]
 \sum\limits_{i=1}^{n} \sum\limits_{j=d+1}^{D} \Big[ -\frac{1}{2\sigma_j^2}z_{ij}^2 \Big]  \\[6pt]
= \ln \bigg( \Big( c_1|\det(W)| \Big)^{n} \Big( \prod\limits_{j=1}^{d} (1+\tau_j) \Big)^{-n} \Big( \prod\limits_{j=1}^{D} \sigma_j \Big)^{-n} \bigg)  - \\[6pt]
  \frac{1}{2} \sum\limits_{j=1}^{d} \Big( \sigma_j^{-2} \sum\limits_{i \in I_{j}}    z_{ij}^2   + \frac{\sigma_j^{-2}}{\tau_{j}^{2} }  \sum\limits_{i \in I_{j}^{c}}   z_{ij}^2  \Big) -  %\\[6pt]
  \frac{1}{2} \sum\limits_{j=d+1}^{D} \sigma_j^{-2} \Big( \sum\limits_{i \in I_{j}}    z_{ij}^2   +  \sum\limits_{i \in I_{j}^{c}}   z_{ij}^2  \Big) \\[6pt]
= \ln \bigg( \Big( c_1|\det(W)| \Big)^{n} \Big( \prod\limits_{j=1}^{d} (1+\tau_j) \Big)^{-n} \Big( \prod\limits_{j=1}^{D} \sigma_j \Big)^{-n} \bigg) - \\[6pt]
 \sum\limits_{j=1}^{d} \frac{1}{2\sigma_j^{2}} \Big(  s_{1j}  + \frac{1}{\tau_{j}^{2} }  s_{2j}  \Big) - \sum\limits_{j=d+1}^{D} \frac{1}{2\sigma_j^{2}} \Big(  s_{1j}  + s_{2j}  \Big).
\end{array}
$$

We fix  $\m$, $W$ and maximize the log-likelihood function over $\tau$ and $\sigma$.
In such a case \textbf{zamiast In such a case uzyj Consequently} we have to \textbf{zamiast we have to uzyj we need to} solve the following system of equations
$$
\begin{array}{l}
\frac{\partial  \ln ( L(X;\m,W,\sigma,\tau) ) }{\partial \sigma_j} =0, \qquad %\\[6pt] & \mbox{ for }  & j=1,\ldots,d,
 \frac{\partial  \ln ( L(X;\m,W,\sigma,\tau) ) }{\partial \tau_j} =0 , %& \mbox{ for }  & j=1,\ldots,d.
\end{array}
$$
for  $ j=1,\ldots,D$. Hence \textbf{It follows that zamiast Hence}
$$
\begin{array}{l}
-\frac{n}{\sigma_j} +  \sigma_j^{-3} (s_{1j} + \tau_j^{-2} s_{2j} ) = 0, \mbox{ for }  j=1,\ldots,d,\\[6pt]
 -\frac{n}{\sigma_j} +  \sigma_j^{-3} (s_{1j} + s_{2j} ) =0, \mbox{ for } j>d, \\[6pt] %
- \frac{n}{1+\tau_j} + \frac{s_{2j}}{\tau_j^{3}\sigma_j^{2}} =0 , \mbox{ for } j=1,\ldots,d.
\end{array}
$$
By simple calculations we obtain the expressions for the estimators in \ref{eq:est}.
%\begin{align*}
%$$
%\hat{\sigma}_j^2(\m,W) = 
%\tfrac{1}{n} s_{1j}^{2/3} g_{j}(\m,W), \qquad
%\hat{\tau}_{j}(\m,W) = \bigg( \frac{s_{2j}}{s_{1j}} \bigg)^{1/3}.
%$$
%\end{align*}
Substituting it into the log-likelihood function,
%and taking $e^{\ln \hat{L}(\m,W)}$
we get %$$
$$
\begin{array}{l}
\hat{L}(\m,W) = \bigg( \frac{2}{\pi} \bigg)^{\frac{dn}{2}}  \bigg( \frac{1}{2\pi} \bigg)^{\frac{(D-d)n}{2}} \!\!\!\!\! |\det(W)|^{n} \Big( \prod\limits_{j=1}^{d} \frac{1}{\sqrt{n}} g_j(\m,W)^{\frac{3}{2}} \Big)^{-n} \!\!\!\!\!
e^{-\frac{dn}{2}} \Big( \prod\limits_{j=d+1}^{D} (\frac{s_{1j} + s_{2j}}{n})^{\frac{1}{2}} \Big)^{-n} \!\!\!\!\! = \\[6pt]
\frac{ 2^{(d-D/2)n} n^{dn/2} }{(\pi e)^{Dn/2}} \cdot %\\[6pt]
\Big( \frac{1}{|\det(W)|^{\frac{2}{3}}} \prod\limits_{j=1}^{d} g_j(\m,W) \Big)^{-\frac{3n}{2}} \Big( \prod\limits_{j=d+1}^{D} \frac{(s_{1j} + s_{2j})}{n} \Big)^{-\frac{n}{2}} 
\end{array}
$$
%= \bigg( \frac{2n}{\pi e} \bigg)^{\frac{dn}{2}} =
\end{proof}

%%%%%%%%%%%%%%%%%%%%%%%%%%%%%%%%


%%%%%%%%%%%%%%%%%%%%%%%%%%%%%%%%
%%%%%%%%%%%%%%%%%%%%%%%%%%%%%%%%
\section{Appendix B}
\label{a2}
%%%%%%%%%%%%%%%%%%%%%%%%%%%%%%%%

We will need the following well-known lemma (for the convenience of the reader we provide the proof).

\begin{lemma}\label{jacobi}
%\comment{Jacek mowi, ze powino być A(t), i napisać co to znaczy adj()}
Let $A = (a_{ij})_{1 \leq i,j \leq d}$ be a differentiable map from real numbers to $d \times d$ matrices then
\begin{equation}
\frac{\partial \det(A)}{\partial a_{ij}} = \mathrm{adj}^T(A)_{ij},
\end{equation}
where $\mathrm{adj}(A)$ stands for the adjugate of $A$, i.e. the transpose of the cofactor matrix.
\end{lemma}

\begin{proof}
By the Laplace expansion $\det A = \sum\limits_{j=1}^{d} (-1)^{i+j} a_{ij} M_{ij}$ where $M_{ij}$ is the minor of the entry in the $i$-th row and $j$-th column. Hence
$$\frac{\partial \det A}{\partial a_{ij}} = (-1)^{i+j} M_{ij} = \mathrm{adj}^T(A)_{ij}.$$
\end{proof}

\begin{proof}[Proof of Theorem \ref{ther:grad}.]
Let us start with the partial derivative of $\ln({l})$ with respect to $\m$. We have
$$
\begin{array}{l}
\frac{\partial \ln {l}(X;\m,W)}{\partial \m_k} =
\sum \limits_{j=1}^d \frac{\partial \ln ({g}_j(\m,W))}{\partial \m_k} + \sum \limits_{j=d+1}^D \frac{\partial \ln ((s_{1j}+s_{2j})^{\frac{1}{3}})}{\partial \m_k} \\[6pt]
= \sum\limits_{j=1}^d \frac{1}{{s}_{1j}^{\frac{1}{3}} + {s}_{2j}^{\frac{1}{3}}} \frac{\partial ({s}_{1j}^{\frac{1}{3}} + {s}_{2j}^{\frac{1}{3}})}{\partial \m_k} + \sum\limits_{j=d+1}^D \frac{1}{({s}_{1j} + {s}_{2j})^{\frac{1}{3}}} \frac{\partial (({s}_{1j} + {s}_{2j})^{\frac{1}{3}})}{\partial \m_k} \\[6pt]
= \sum \limits_{j=1}^d \frac{1}{{s}_{1j}^{\frac{1}{3}} + {s}_{2j}^{\frac{1}{3}}} \bigg(
\frac{1}{3 {s}_{1j}^{\frac{2}{3}}} \frac{\partial {s}_{1j}}{\partial \m_k} +
\frac{1}{3 {s}_{2j}^{\frac{2}{3}}} \frac{\partial {s}_{2j}}{\partial \m_k}
\bigg) \\[6pt]
+ \sum \limits_{j=d+1}^D \frac{1}{({s}_{1j} + {s}_{2j})^{\frac{1}{3}}} \frac{1}{3} \frac{1}{({s}_{1j} + {s}_{2j})^{\frac{2}{3}}}\bigg(
\frac{\partial {s}_{1j}}{\partial \m_k} +
\frac{\partial {s}_{2j}}{\partial \m_k}
\bigg).
\end{array}
$$
Now, we need $\frac{\partial {s}_{1j}}{\partial \m_k}$ and $\frac{\partial {s}_{2j}}{\partial \m_k}$, therefore
$$
\begin{array}{l}
\frac{\partial {s}_{1j}}{\partial \m_k} = 
\sum\limits_{i \in {I}_j} \frac{\partial [\w^T_j (\x_i - \m)]^2}{\partial \m_k} =\\[6pt]
 \sum\limits_{i \in {I}_j} 2 \w^T_j (\x_i - \m) \frac{\partial \w^T_j (\x_i - \m)}{\partial \m_k} = %\\[6pt]
 \sum\limits_{i \in {I}_j} - 2 \w^T_j (\x_i - \m) \w_{jk}.
\end{array}
$$
Analogously we get
$$
\begin{array}{l}
\frac{\partial {s}_{2j}}{\partial \m_k} = \sum\limits_{i \in {I}_j^c} -2 \w^T_j (\x_i - \m) \w_{jk}.
\end{array}
$$
%\comment{$\v^{-1}_{jk} = \w_{jk}$}\\
Hence 
$$
\begin{array}{l}
\frac{\partial \ln {l}}{\partial \m_k} =\sum\limits_{j=1}^d \frac{-1}{{s}_{1j}^{\frac{1}{3}} + {s}_{2j}^{\frac{1}{3}}} \bigg(
\frac{1}{3 {s}_{1j}^{\frac{2}{3}}} \sum\limits_{i \in I_j} 2 \w_j^T (\x_i - \m)  \w_{jk} + \\[6pt]
\frac{1}{3 {s}_{2j}^{\frac{2}{3}}} \sum\limits_{i \in I_j^c} 2 \w_j^T (\x_i - \m) \w_{jk}
\bigg) + \sum\limits_{j=d+1}^D \frac{-1}{3(s_{1j}+s_{2j})} \cdot \\[6pt]
\bigg(
 \sum\limits_{i \in I_j} 2 \w_j^T (\x_i - \m)  \w_{jk} +% \\[6pt]
 \sum\limits_{i \in I_j^c} 2 \w_j^T (\x_i - \m) \w_{jk}
\bigg)
.
\end{array}
$$

Now we calculate the partial derivative of $\ln {l}(X;\m,W)$ with respect to the matrix $W$. We have $\frac{\partial \ln {l}(X;\m,W)}{\partial \w_{pk}} =$
$$
\begin{array}{l}
\frac{\partial \ln |\det(W)|^{-\frac{2}{3}}}{\partial \w_{pk}} + \sum\limits_{j=1}^d \frac{\partial \ln ({g}_j(\m,W))}{\partial \w_{pk}}
+ \sum\limits_{j=d+1}^D \frac{\partial \ln ( (s_{1j}+s_{2j})^{\frac{1}{3}} )}{\partial \w_{pk}}.
\end{array}
$$
%\comment{$\v_{pk}^{-1} = \w_{pk}$}\\
To calculate the derivative of the determinant we use Jacobi's formula (see Lemma \ref{jacobi}).
Hence% $\frac{\partial \ln (\det(W)^{-\frac{2}{3}})}{\partial \w_{pk}} =$
$$
\begin{array}{l}
\frac{\partial \ln (\det(W)^{-\frac{2}{3}})}{\partial \w_{pk}} = \det(W)^{\frac{2}{3}}  \Big(-\frac{2}{3}\Big)  \det(W)^{-\frac{5}{3}}  \frac{\partial \det(W)}{\partial \w_{pk}} \\[6pt]
= -\frac{2}{3} \det(W)^{-1}  \mathrm{adj}^T(W)_{pk} \\[6pt]
 = -\frac{2}{3} \frac{1}{\det(W)}  \left[\det(W)  (W^{-1})^T_{pk}\right]= -\frac{2}{3}  (\w^{-1})^T_{pk},
\end{array}
$$
where $(\w^{-1})^T_{pk}$ is the element in the $p$-th row and $k$-th column of the matrix $(W^{-1})^T$. Now we calculate %$\frac{\partial \ln ({g}_j(\m,W))}{\partial \w_{pk}} =$
$$
\begin{array}{l}
\frac{\partial \ln ({g}_j(\m,W))}{\partial \w_{pk}} = \frac{1}{{s}_{1j}^{\frac{1}{3}} + {s}_{2j}^{\frac{1}{3}}} \frac{\partial ({s}_{1j}^{\frac{1}{3}} + {s}_{2j}^{\frac{1}{3}})}{\partial \w_{pk}}= \\[6pt]
\frac{1}{{s}_{1j}^{\frac{1}{3}} + {s}_{2j}^{\frac{1}{3}}} \bigg(
\frac{1}{3 {s}_{1j}^{\frac{2}{3}}}  \frac{\partial {s}_{1j}}{\partial \w_{pk}} +
\frac{1}{3 {s}_{2j}^{\frac{2}{3}}}  \frac{\partial {s}_{2j}}{\partial \w_{pk}}
\bigg),
\end{array}
$$
where
$$
\begin{array}{l}
\frac{\partial {s}_{1j}}{\partial \w_{pk}} = \sum\limits_{ i \in {I}_j} \frac{\partial [\w^T_j (\x_i - \m)]^2}{\partial \w_{pk}} = \sum\limits_{ i \in {I}_j} 2 \w^T_j (\x_i - \m) \frac{\partial \w^T_j (\x_i - \m)}{\partial \w_{pk}}
\\[6pt]
= \left\{ \begin{array}{ll}
0, & \text{if} \; j\neq p\\
\sum\limits_{ i \in {I}_p} 2 \w^T_p (\x_i - \m) (\x_{ik} - \m_k), & \text{if} \; j=p\\
\end{array} \right.
\end{array}
$$
and $\x_{ik}$ is the $k$-th element of the vector $\x_i$. Analogously we get
$$\frac{\partial {s}_{2j}}{\partial \w_{pk}} = \left\{ \begin{array}{ll}
0, & \text{if} \; j\neq p\\
\sum\limits_{ i \in {I}_p^c} 2 \w^T_p (\x_i - \m) (\x_{ik} - \m_k), & \text{if} \; j=p.
\end{array} \right.
$$
Moreover,
$$
\begin{array}{l}
\frac{\partial \ln ( (s_{1j}+s_{2j})^{\frac{1}{3}} )}{\partial \w_{pk}} = \frac{1}{ (s_{1j}+s_{2j})^{\frac{1}{3}} } \frac{\partial ( (s_{1j}+s_{2j})^{\frac{1}{3}} )}{\partial \w_{pk}}= \\[6pt]
\frac{1}{ (s_{1j}+s_{2j})^{\frac{1}{3}} } \frac{1}{3} \frac{1}{ (s_{1j}+s_{2j})^{\frac{2}{3}} } \bigg(
  \frac{\partial {s}_{1j}}{\partial \w_{pk}} +
  \frac{\partial {s}_{2j}}{\partial \w_{pk}}
\bigg),
\end{array}
$$
Hence we obtain
$$
\begin{array}{l}
\frac{\partial \ln {l}}{\partial \w_{pk}} = -\frac{2}{3} (\w^{-1})^T_{pk} + \\[6pt]
\frac{1}{{s}_{1p}^{\frac{1}{3}} +{s}_{2p}^{\frac{1}{3}}} 
 \bigg(
\frac{1}{3} {s}_{1p}^{-\frac{2}{3}} \sum\limits_{ i \in {I}_p} 2 \w^T_p (\x_i - \m) (\x_{ik} - \m_k)\\[6pt]
+ \frac{1}{3} {s}_{2p}^{-\frac{2}{3}} \sum\limits_{ i \in {I}_p^c} 2 \w^T_p (\x_i - \m) (\x_{ik} - \m_k) \bigg)+\\[6pt]

\frac{1}{ 3(s_{1p}+s_{2p}) } 
 \bigg(
\sum\limits_{ i \in {I}_p} 2 \w^T_p (\x_i - \m) (\x_{ik} - \m_k) + \\[6pt]
\sum\limits_{ i \in {I}_p^c} 2 \w^T_p (\x_i - \m) (\x_{ik} - \m_k) \bigg)
.
\end{array}
$$

\end{proof}
%%%%%%%%%%%%%%%%%%%%%%%%%%%%%%%%

%% Acknowledgements should only appear in the accepted version. 
%\section*{Acknowledgements} 
% 
%\textbf{Do not} include acknowledgements in the initial version of
%the paper submitted for blind review.
%
%If a paper is accepted, the final camera-ready version can (and
%probably should) include acknowledgements. In this case, please
%place such acknowledgements in an unnumbered section at the
%end of the paper. Typically, this will include thanks to reviewers
%who gave useful comments, to colleagues who contributed to the ideas, 
%and to funding agencies and corporate sponsors that provided financial 
%support.  


% In the unusual situation where you want a paper to appear in the
% references without citing it in the main text, use \nocite

%\bibliographystyle{icml2017}
\bibliography{ref}
\bibliographystyle{icml2017}

\end{document} 


%%%%%%%%%%%%%%%%%%%%%%%%%%%%%%%%%%%%%%%%%%%%


@article{wang2015abundance,
  title={An abundance characteristic-based independent component analysis for hyperspectral unmixing},
  author={Wang, Nan and Du, Bo and Zhang, Liangpei and Zhang, Lifu},
  journal={IEEE Transactions on Geoscience and Remote Sensing},
  volume={53},
  number={1},
  pages={416--428},
  year={2015},
  publisher={IEEE}
}


@article{caiafa2008blind,
  title={Blind spectral unmixing by local maximization of non-Gaussianity},
  author={Caiafa, Cesar F and Salerno, Emanuele and Proto, Araceli N and Fiumi, L},
  journal={Signal Processing},
  volume={88},
  number={1},
  pages={50--68},
  year={2008},
  publisher={Elsevier}
}


%%%%%%%%%%%%%%%%%%%%%%%%%%%%%%%%%%%%%%%%%%%%




% This document was modified from the file originally made available by
% Pat Langley and Andrea Danyluk for ICML-2K. This version was
% created by Lise Getoor and Tobias Scheffer, it was slightly modified  
% from the 2010 version by Thorsten Joachims & Johannes Fuernkranz, 
% slightly modified from the 2009 version by Kiri Wagstaff and 
% Sam Roweis's 2008 version, which is slightly modified from 
% Prasad Tadepalli's 2007 version which is a lightly 
% changed version of the previous year's version by Andrew Moore, 
% which was in turn edited from those of Kristian Kersting and 
% Codrina Lauth. Alex Smola contributed to the algorithmic style files.  

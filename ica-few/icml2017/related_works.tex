%%%%%%%%%%%%%%%%%%%%%%%%%%%%%%%%
\section{Related works}
\label{related_works}
%%%%%%%%%%%%%%%%%%%%%%%%%%%%%%%%

In literature exist a few approaches dedicated for non-square ICA problem.
Most of such method are dedicated for individual methods.
Attias and Schreiner \cite{attias1997blind} derived a likelihood based algorithm for separation of general sequences with a frequency domain implementation.
Belouchrani and Cardoso \cite{belouchrani1995maximum} presented a general likelihood approach allowing
for additive noise and for non-square mixing matrices. They applied the
method to separation of sources taking discrete values and estimated the
mixing matrix using an Expectation-Maximization (EM) approach with both a deterministic and a stochastic formulation. In \cite{moulines1997maximum} authors used the EM approach
for separation of autocorrelated sequences in presence of noise and explored
a family of flexible source signal priors based on Gaussian Mixtures. 

The assumption is that of square mixing is mostly unrealistic in  the case of EEG ant FMRI where the number of sources is less than the number of electrodes \cite{beckmann2004probabilistic,samarov2004nonparametric,shi2017investigating}. Therefore, many of algorithms dedicated for this task use a probabilistic ICA \cite{tipping1999probabilistic}.
The noisy ICA model can be approximated using a variant of PCA+ICA \cite{beckmann2004probabilistic}, where probabilistic PCA is used to estimate the number of components and achieve dimension reduction \cite{tipping1999probabilistic}.
In \cite{allassonniere2012stochastic} authors developed stochastic EM algorithms to estimate the noisy model and proposed parametric methods.

Other methods exploring non-Gaussian structure in multivariate data include non-Gaussian component analysis (NGCA) and projection pursuit \cite{blanchard2006search,kawanabe2007new}.
NGCA is a more general case of linear non-Gaussian component analysis (LNGCA) \cite{risk2015likelihood}
that allows non-linear dependence between the non-Gaussian components.

In the paper  \cite{miettinen2014deflation} authors propose a novel adaptive twostage deflation-based FastICA algorithm that allows one to use different nonlinearities for different components and optimizes the order in which the components are extracted.








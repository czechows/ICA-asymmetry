%%%%%%%%%%%%%%%%%%%%%%%%%%%%%%%%%%%%%%%%%%%%%%%%%%%%%%%%%%%%%%%%%%
%%%%%%%% ICML 2016 EXAMPLE LATEX SUBMISSION FILE %%%%%%%%%%%%%%%%%
%%%%%%%%%%%%%%%%%%%%%%%%%%%%%%%%%%%%%%%%%%%%%%%%%%%%%%%%%%%%%%%%%%

% Use the following line _only_ if you're still using LaTeX 2.09.
%\documentstyle[icml2016,epsf,natbib]{article}
% If you rely on Latex2e packages, like most moden people use this:
\documentclass{article}

\usepackage{amsmath,amsfonts,amssymb,amsthm,dsfont}
% use Times
\usepackage{times}
% For figures
\usepackage{graphicx} % more modern
%\usepackage{epsfig} % less modern
\usepackage{subfigure} 

% For citations
\usepackage{natbib}

% For algorithms
\usepackage{algorithm}
\usepackage{algorithmic}

% As of 2011, we use the hyperref package to produce hyperlinks in the
% resulting PDF.  If this breaks your system, please commend out the
% following usepackage line and replace \usepackage{icml2016} with
% \usepackage[nohyperref]{icml2016} above.
\usepackage{hyperref}

% Packages hyperref and algorithmic misbehave sometimes.  We can fix
% this with the following command.
\newcommand{\theHalgorithm}{\arabic{algorithm}}

% Employ the following version of the ``usepackage'' statement for
% submitting the draft version of the paper for review.  This will set
% the note in the first column to ``Under review.  Do not distribute.''
\usepackage{icml2016} 

\def\N{\mathbb{N}}
\def\Z{\mathbb{Z}}
\def\R{\mathbb{R}}
\def\C{\mathcal{C}}

\def\e{\varepsilon}
\def\d{\delta}
%\def\w{\omega}
\def\w{\mathrm{w}}
\def\v{\mathrm{V}}
\def\x{\mathrm{x}}
\def\m{\mathrm{m}}
\def\z{\mathrm{z}}

\def\F{\mathcal{F}}
\def\G{\mathcal{G}}
\def\S{\mathcal{S}}
\def\A{\mathcal{A}}
\def\M{\mathcal{M}}

\def\1{\mathds{1}}

\def\KL{\mathrm{KL}}

\def\for{\mbox{  for }}
\def\mle{\mathrm{mle}}
\def\diag{\mathrm{diag}}
\def\cov{\mathrm{cov}}
\def\card{\mathrm{card}}
\def\aff{\mathrm{aff}}
\def\span{\mathrm{span}}
\def\nor{\mathcal{N}}
\DeclareMathOperator*{\argmin}{argmin}

\newtheorem{observation}{Observation}[section]
\newtheorem{theorem}{Theorem}[section]
\newtheorem{proposition}{Proposition}[section]
\newtheorem{lemma}{Lemma}[section]
\newtheorem{corollary}{Corollary}[section]

\theoremstyle{definition}
\newtheorem{definition}{Definition}[section]
\newtheorem{remark}{Remark}[section]
\newtheorem{example}{Example}[section]
\newtheorem{problem}{Problem}[section]


% Employ this version of the ``usepackage'' statement after the paper has
% been accepted, when creating the final version.  This will set the
% note in the first column to ``Proceedings of the...''
%\usepackage[accepted]{icml2016}


% The \icmltitle you define below is probably too long as a header.
% Therefore, a short form for the running title is supplied here:
\icmltitlerunning{Submission and Formatting Instructions for ICML 2016}

\begin{document} 

\twocolumn[
\icmltitle{Submission and Formatting Instructions for \\ 
           International Conference on Machine Learning (ICML 2016)}

% It is OKAY to include author information, even for blind
% submissions: the style file will automatically remove it for you
% unless you've provided the [accepted] option to the icml2016
% package.
\icmlauthor{Your Name}{email@yourdomain.edu}
\icmladdress{Your Fantastic Institute,
            314159 Pi St., Palo Alto, CA 94306 USA}
\icmlauthor{Your CoAuthor's Name}{email@coauthordomain.edu}
\icmladdress{Their Fantastic Institute,
            27182 Exp St., Toronto, ON M6H 2T1 CANADA}

% You may provide any keywords that you 
% find helpful for describing your paper; these are used to populate 
% the "keywords" metadata in the PDF but will not be shown in the document
\icmlkeywords{boring formatting information, machine learning, ICML}

\vskip 0.3in
]

\begin{abstract} 
Independent Component Analysis (ICA) - one of the basic tools in data analysis - aims to find a coordinate system in which the components of the data are independent. In many application the number of sources is unknown and may be less than the number of sensors. In such situation we are looking for so-called non-square mixing matrix.
Due to computational constraints, principal component analysis is used for dimension reduction prior to ICA (PCA+ICA),
which could remove important information.
In this paper we present a two method which are dedicated for determining non-square mixing matrix by fitting non Gaussian densities.
\end{abstract} 



%%%%%%%%%%%%%%%%%%%%%%%%%%%%%%%%
\section{Introduction}
\label{introduction}
%%%%%%%%%%%%%%%%%%%%%%%%%%%%%%%%%%%%%%%%%%%%%%%%%%%%%%%%%%%%%%%%
ICA is similar in many aspects to principal component analysis (PCA). In PCA we look for an orthonormal change of basis so that the components are not
linearly dependent (uncorrelated).
ICA can be described as a search for the optimal basis (coordinate system) in which the components are independent. Let us now, for the readers convenience, describe how the 
ICA works. The data are represented by the random vector $\x$
 and the components as the random vector~$s$.  Our aim is to transform the observed data $\x$ into maximally independent components $s$ with respect to some measure  
of independence. Here we use a linear static transformation $W$, called the {\em transformation matrix}, combined with the formula 
$$
s = W \x.
$$

Popular ICA methodology does not directly attempt to find components that are independent but rather components that are as non-Gaussian as possible.
This follows from the fact that one of the theoretical foundations of ICA is given by the dual view at the Central Limit Theorem \cite{hyvarinen2000independent}, which states that the distribution of the sum (average or linear combination) of $N$ independent random variables approaches Gaussian as  $N\rightarrow \infty$. Obviously if all source variables are Gaussian, the ICA method will not work. 

There exists many different approaches to ICA which uses negentropy \cite{hyvarinen2000independent}, cumulant-based methods \cite{cardoso1993blind,virta2015joint}, maximum likelihood methods \cite{chen2006efficient,samworth2012independent} and methods that directly minimize a measure of dependence \cite{stogbauer2004least,matteson2016independent}.

In many application the number of sources is unknown and may be less than the number of sensors. In such situation we are looking for so-called non-square mixing matrix.
In practice, PCA is applied to the observations prior to classic ICA (PCA+ICA) to meet the assumption of square mixing and to reduce computational
costs \cite{hyvarinen2004independent}. PCA+ICA is commonly used to identify brain networks
in functional magnetic resonance imaging (fMRI) \cite{beckmann2012modelling,green2002pca} and hyperspectral unmixing \cite{wang2015abundance,caiafa2008blind}.

The problem in such approach is that interesting independent components (ICs) could be mixed in several principal components that are discarded and then these ICs cannot be recovered.

In the paper we present two methods dedicated to a maximum-likelihood framework. In the firs case we are looking directly $d \leq D$ independent component by maximization of likelihood function. The second method work in full dimensional space by  estimating density congaing $d$ non-gaussian components (independent ones) and $D-d$ gaussian ones which model a noise.  

[!!!Opisac w miare dokladnie nasze podejscie!!!]



%%%%%%%%%%%%%%%%%%%%%%%%%%%%%%%%
%%%%%%%%%%%%%%%%%%%%%%%%%%%%%%%%
 
%%%%%%%%%%%%%%%%%%%%%%%%%%%%%%%%
\section{basic tools}
\label{sec_1}
%%%%%%%%%%%%%%%%%%%%%%%%%%%%%%%%%%%%%%%%%%%%%%%%%%%%%%%%%%%%%%%%


\subsection{Measure of nongaussianity}

We consider the similar idea to the Kullback-Leibler.

\subsection{Construction of densities}

We can define the family of singular densities on affine subspaces of 
dimension $d$, by taking the transport.


In this subsection we describe the basic construction of product measures and densities. Given functions $f_1,f_2$ on $\R^{d_1},\R^{d_2}$ by 
$$
(f_1 \otimes f_2)(x_1,x_2)=f_1(x_1) \cdot f_2(x_2) \for (x_1,x_2) \in \R^{d_1} 
\times \R^{d_2}
$$ 
we denote the tensor
product of $f_1$ and $f_2$. Observe that if $f_1,f_2$ are densities, then so is
$f_1 \otimes f_2$.

If $\F$ is a family of densities on $\R$, then by $\F^{\otimes d}$ ($d$-th tensor power of $\F$) we denote
the family of densities on $\R^d$ given by
$$
\F^{\otimes k}=\{f_1 \otimes \ldots \otimes f_d : f_i \in \F\}.
$$




\subsection{Our case}

We assume that we have a family $\F$ larger then Gaussians on $\R$.

We have
$$
\KL(X,\aff(\S^{\otimes d}),\G^d)=\inf_{m,\v} \KL((v^T\v)^{-1}\v^T(X-m),\S^{\otimes d},\G).
$$
Notation: $x[m,\v]$. By the $i$-th coordinate we denote $x[m,\v]_i$.

Thus 
$$
\KL(X,\aff(\S^{\otimes d}),\G^d)=\inf_{m,\v} \left( \sum_{i=1}^d \mle(X[m,\v]_i,\S)
-\mle(X[m,\v],\G)
\right),
$$
where the minus has the direct formula which can be computed.

%%%%%%%%%%%%%%%%%%%%%%%%%%%%%%%%%%%%%%%%%%
\subsection{First approach: global estimation}

We search for the split $g=f \otimes \N$, where $g$ is a normal density on $\R^{D-d}$,
and $f$ is $d$-dimensional. More precisely, we fix a family $\F$ of densities on $\R^d$, and seek $m,\v$ which maximize the MLE:
$$
X \sim a_*(f \otimes g) \text{ for } f \in \F, g \in \nor(\R^{D-d}), a=a_{m,\v} \in \aff(\R^D). 
$$

In the case when $\F$ is one dimensional, the above can be written as:
$$
X \sim \det W \cdot f_1(w_1 \circ (x-m)) \cdot \ldots \cdot f_d(w_d \circ (x-m))
\cdot g_{d+1}(w_{d+1} \circ (x-m)) \cdot \ldots \cdot g_D(w_D \circ (x-m))
$$
where $W=[w_1,\ldots,w_D]=??(V^{-1})^T$.

%%%%%%%%%%%%%%%%%%%%%%%%%%%%%%%%%%%%%%%%%%
\subsection{Number of coordinates}

poszukac - oni cos pisza o ilosci wspolrzednych



%%%%%%%%%%%%%%%%%%%%%%%%%%%%%%%%
%%%%%%%%%%%%%%%%%%%%%%%%%%%%%%%% 
 
%%%%%%%%%%%%%%%%%%%%%%%%%%%%%%%%
\section{przemek}
\label{p}
%%%%%%%%%%%%%%%%%%%%%%%%%%%%%%%%%%%%%%%%%%%%%%%%%%%%%%%%%%%%%%%%


 The density of the one-dimensional Split Gaussian distribution is given by the formula
$$
SN(x;m,\sigma^2,\tau^2) = \left\{ \begin{array}{ll}
c \cdot \exp[-\frac{1}{2\sigma^2}(x-m)^2], & \textrm{where $x\leq m$}\\
c \cdot \exp[-\frac{1}{2\tau^2\sigma^2}(x-m)^2], & \textrm{where $x>m$}\\
\end{array} \right.
$$
where $c=\sqrt{\frac{2}{\pi}}\sigma^{-1}(1+\tau)^{-1}$. 

A natural generalization of the univariate split normal distribution to the multivariate settings was presented by \cite{villani2006multivariate}.
%\comment{(\cite{john1982three})}
Roughly speaking, authors assume that a vector $\x \in \R^d$ follows the multivariate Split Normal distribution, if its principal components are orthogonal and follow the one-dimensional Split Normal distribution.

\begin{definition}\label{def:SN}
A density of the multivariate Split Normal distribution is given by
$$
 SN_{d}(\x; \m, \sigma,\tau)= \prod_{j=1}^{d} SN(x_j;m_j,\sigma_j^2,\tau_j^2),
$$
where  $\m = [m_1, \ldots, m_d]^T$, $\sigma = [\sigma_{1}^2,\ldots,\sigma_{d}^2]^T$ and $\tau=[\tau_{1}^2,\ldots,\tau_{d}^2]^T$.
%where $W$ is the orthonormal matrix and $\w_{j}$ stand for the $j$-th column of $W$, $\m = (m_1, \ldots, m_d)^T$, $\sigma = (\sigma_{1},\ldots,\sigma_{d})$ and $\tau=(\tau_{1},\ldots,\tau_{d})$.
\end{definition}


In our case we will use density on projection on $d<D$ subspaces. Therefore we need a density $d$-subspace Split Normal distribution.

\begin{definition}\label{def:GSN}
A density of the multivariate $d$-subspace Split Normal distribution is given by
$$
 SN_{d<D}(\x; \m,W, \sigma^2,\tau^2)=  SN_d((W^TW)^{-1}W^T(\x-\m);0,\sigma^2,\tau^2),
$$
where
%%%%%%%%%
$(W^TW)^{-1}W^T(x-m) \in \R^d$
%%%%%%%%%
 $\w_{j} \in \R^D$ is the $j$-th column of non-singular matrix $W = [w_{1},\ldots,w_{d}]$, $\m = [m_1, \ldots, m_D]^T$, $\sigma = [\sigma_{1},\ldots,\sigma_{d}]^T$ and $\tau=[\tau_{1},\ldots,\tau_{d}]^T$.
\end{definition}

Let us recall that the standard Gaussian density in $\R^d$ is defined by 
$$
N(\x;\m,\Sigma)=\frac{1}{(2\pi)^{d/2} \det(\Sigma)^{1/2}} \exp \left(-\tfrac{1}{2} (\x-\m)^T \Sigma^{-1}(\x-\m) \right),
$$
where $\m$ denotes the mean, $\Sigma$ is the covariance matrix.

\begin{definition}\label{def:GSN}
A density of the multivariate $d$-subspace Normal distribution is given by
$$
 N_{d<D}(\x; \m, \Sigma, W)= N((W^TW)^{-1}W^T(\x-\m);0,\Sigma),
$$
where
%%%%%%%%%
$(W^TW)^{-1}W^T(\x-\m) \in \R^d$
%%%%%%%%%
 $\w_{j} \in \R^D$ is the $j$-th column of non-singular matrix $W = [w_{1},\ldots,w_{d}]$, $\m = [m_1, \ldots, m_D]^T$, $\Sigma = \diag(\sigma_{1}^2,\ldots,\sigma_{d}^2)$.
\end{definition}

Our goal is to minimize
$$
\KL(X,\F,\G)=\mle(X,\F)-\mle(X,\G) 
$$
In our language
\begin{equation}
\begin{array}{l}
\KL_{d<D}(X;\m,W,\sigma,\tau,\Sigma) = \\[6pt]
= \sum \limits_{\x \in X} \ln(SN_{d<D}(\x;\m,W,\sigma,\tau)) -
   \sum \limits_{\x \in X} \ln(N_{d<D}(\x;\m,\Sigma,W))
\end{array}
\end{equation}
We known
$$
\sum \limits_{\x \in X} \ln(N_{d<D}(\x;\m,\Sigma,W)) = -\frac{d}{2}\ln(2\pi e)-\frac{1}{2}\ln \det(\Sigma_{W}), 
$$
where 
$$
\Sigma_{W} = \cov( \{ (W^TW)^{-1}W^T(\x-\m) \colon \x \in \R^D\} )
$$

%%%%%%%%%%%%%%%%%%%
\subsection{Optimization problem}
%%%%%%%%%%%%%%%%%%%

The density of the multivariate d-subspace Normal distribution depends on four parameters $\m \in \R^d$, $W \in \M(\R^d)$, $\sigma \in \R^d$, $\tau \in \R^d$. 
We can find them by minimizing the simpler function, which depends on only  $m \in \R^d$ and $W \in \M(\R^d)$. Other parameters are given by explicit formulas. Let us notice that in this case our minimization problem simplifies to minimizing the function $\mle(X,\F) = \sum \limits_{\x \in X} \ln(SN_{d<D}(\x;\m,W,\sigma,\tau))$

\begin{theorem}\label{the:min}
Let $\x_1,\ldots,\x_n$ be given.  
Then the likelihood maximized w.r.t. $\sigma$ and $\tau$ is
\begin{equation}\label{eq:1}
%\min_{\sigma, \tau}
 \hat{L}(X;\m,W) =   \bigg( \frac{2n}{\pi e} \bigg)^{dn/2} \bigg( \prod_{j=1}^{d} g_{j}(\m,W) \bigg)^{-3n/2},
\end{equation}
where
$$
\begin{array}{c}
{g}_{j}(\m,W) = {s}_{1j}^{1/3} + {s}_{2j}^{1/3},
%\\[1ex]
%W_{\omega}=(W^TW)^{-1}W^T,
\\[1ex]
{s}_{1j}= \! \sum\limits_{i \in I_j}[ \w_{j}^T (\x_i-\m)]^2,  {I}_j=\{ i = 1,\ldots,n \colon \w_{j}^T (\x_i-\m) \leq 0 \},
\\[1ex]
{s}_{2j}= \! \sum\limits_{i \in I_j^c}[ \w_{j}^T (\x_i-\m)]^2, {I}_j^c=\{ i = 1,\ldots,n \colon  \w_{j}^T (\x_i-\m) > 0 \},
\end{array}
$$
where $\omega_j$ is the $j$-th column of non-singular matrix $(W^TW)^{-1}W^T$ and the maximum likelihood estimators of $\sigma_{j}^2$ and $\tau_{j}$ are
$$\hat \sigma_j^2(\m,W) = \tfrac{1}{n} s_{1j}^{2/3} g_{j}(\m,W), \quad
\hat \tau_{j}(\m,W)=\left(\frac{s_{2j}}{s_{1j}}\right)^{1/3}.
$$
\end{theorem}

\begin{proof}[Proof of Theorem \ref{the:min}.]
Let $X=\{ \x_1, \ldots, \x_n \}$ and $W_{\omega}=(W^TW)^{-1}W^T$.
We write 
$$
\z_i=  W_{\omega}(\x_i-m), \quad \z_{ij}= \omega_j^T(\x_i-m),
$$
for observation $i$, where $i=1,\ldots,n$ and coordinates $j=1,\ldots,d$.

Let us consider the likelihood function, i.e. 
$$
\begin{array}{l}
L(X;\m,W,\sigma,\tau) = \prod\limits_{i=1}^n SN_{d<D}(\x_i;\m,W,\sigma,\tau) =  
%= \sum \limits_{\x \in X} \ln(SN_{d<D}(\x;\m,W,\sigma,\tau)) 
\prod\limits_{i=1}^{n} \prod\limits_{j=1}^{d} SN(\omega_j^T(\x_i - \m) ; 0,\sigma^2,\tau^2)
\\[6pt]
= c_1^n \Big( \prod\limits_{j=1}^{d} \sigma_j(1+\tau_j) \Big)^{-n} %\cdot \\[6pt]
\prod\limits_{i=1}^{n} \prod\limits_{j=1}^{d} \exp \Big[ -\frac{1}{2\sigma_j^2}z_{ij}^2 (\1_{ \{ z_{ij} \leq 0 \} } + \tau_{j}^{-2} \1_{ \{ z_{ij} > 0 \} }) \Big],
\end{array}
$$
%$$
%\begin{array}{l}
%= \prod\limits_{i=1}^{n} GSN_d(\x_i ; \m,V,\sigma,\tau)
%= \prod\limits_{i=1}^{n} \frac{1}{| \det( V)|}  \prod\limits_{j=1}^{d} SN(  \v^{-1}_j \x_i ; \v^{-1}_j \m , \sigma_j^2, \tau_j^2)=
%\\[1ex]
%\left(\frac{c_1}{|\det(V)|}\right)^{n} \left( \prod\limits_{j=1}^{d} \sigma_j(1+\tau_j) \right)^{-n} \left( \prod\limits_{i=1}^{n} \prod\limits_{j=1}^{d} \exp[-\frac{1}{2\sigma_j^2}z_{ij}^2 (\1_{ \{ z_{ij} \leq 0 \} } + \tau_{j}^{-2} \1_{ \{ z_{ij} > 0 \} })]\right)
%\end{array}
%$$
where 
$
c_1=\left( \sqrt{\tfrac{2}{\pi}} \right)^{d}.
$
Now we take the log-likelihood function, i.e.
$$
\begin{array}{l}
\ln(L(X;\m,W,\sigma,\tau)) \\[6pt]
=\ln \bigg( c_1^n \Big( \prod\limits_{j=1}^{d} \sigma_j(1+\tau_j) \Big)^{-n} \bigg) + %\\[6pt]
 \sum\limits_{i=1}^{n} \sum\limits_{j=1}^{d} \Big[ -\frac{1}{2\sigma_j^2}z_{ij}^2 (\1_{ \{ z_{ij} \leq 0 \} } + \tau_{j}^{-2} \1_{ \{ z_{ij} > 0 \} })\Big]  \\[6pt]
= \ln \bigg( c_1^n \Big( \prod\limits_{j=1}^{d} \sigma_j(1+\tau_j) \Big)^{-n} \bigg)  -%\\[6pt]
  \frac{1}{2} \sum\limits_{j=1}^{d} \Big( \sigma_j^{-2} \sum\limits_{i \in I_{j}}    z_{ij}^2   + \frac{\sigma_j^{-2}}{\tau_{j}^{2} }  \sum\limits_{i \in I_{j}^{c}}   z_{ij}^2  \Big) \\[6pt]
= \ln \bigg( c_1^n \Big( \prod\limits_{j=1}^{d} \sigma_j(1+\tau_j) \Big)^{-n} \bigg)  - 
 \sum\limits_{j=1}^{d} \frac{1}{2\sigma_j^{2}} \Big(  s_{1j}  + \frac{1}{\tau_{j}^{2} }  s_{2j}  \Big).
\end{array}
$$

We fix  $\m$, $W$ and maximize the log-likelihood function over $\tau$ and $\sigma$.
In such a case we have to solve the following system of equations
$$
\begin{array}{l}
\frac{\partial  \ln ( L(X;\m,W,\sigma,\tau) ) }{\partial \sigma_j} = -\frac{n}{\sigma_j} +  \sigma_j^{-3} (s_{1j} + \tau_j^{-2} s_{2j} )
 =0, \\[6pt] %& \mbox{ for }  & j=1,\ldots,d,
 \frac{\partial  \ln ( L(X;\m,W,\sigma,\tau) ) }{\partial \tau_j} = - \frac{n}{1+\tau_j} + \frac{s_{2j}}{\tau_j^{3}\sigma_j^{2}} =0 , %& \mbox{ for }  & j=1,\ldots,d.
\end{array}
$$
for  $ j=1,\ldots,d$.
By simple calculations we obtain the expressions for the estimators
%\begin{align*}
$$
\hat{\sigma}_j^2(\m,W) = 
\tfrac{1}{n} s_{1j}^{2/3} g_{j}(\m,W), \qquad
\hat{\tau}_{j}(\m,W) = \bigg( \frac{s_{2j}}{s_{1j}} \bigg)^{1/3}.
$$
%\end{align*}
Substituting it into the log-likelihood function,
%and taking $e^{\ln \hat{L}(\m,W)}$
we get
$$
\begin{array}{l}
\hat{L}(\m,W) = \bigg( \frac{2}{\pi} \bigg)^{\frac{dn}{2}} \Big( \prod\limits_{j=1}^{d} \frac{1}{\sqrt{n}} g_j(\m,W)^{\frac{3}{2}} \Big)^{-n}  e^{-\frac{dn}{2}}\\[6pt]
= \bigg( \frac{2n}{\pi e} \bigg)^{\frac{dn}{2}}  \Big( \prod\limits_{j=1}^{d} g_j(\m,W) \Big)^{-\frac{3n}{2}}. 
\end{array}
$$
\end{proof}

%\begin{theorem}\label{the:Villani}
%Given a random sample $\x_1,\ldots,\x_n$ from $SN(\mu,\lambda^2,\tau^2)$, the likelihood, maximized over $\lambda$ and $\tau$, is
%\begin{equation}\label{eq:Villani}
%%\min_{\sigma, \tau}
% \hat{L}(\mu) =   \bigg( \frac{2n}{\pi e} \bigg)^{n/2} \bigg( \prod_{j=1}^{d} g_{j}(\m,W) \bigg)^{-3n/2},
%\end{equation}
%where
%$$
%\begin{array}{c}
%{g}_{j}(\m,W) = {s}_{1j}^{1/3} + {s}_{2j}^{1/3},
%\\[1ex]
%{s}_{1j}= \! \sum\limits_{i \in I_j}[ \w_{j}^T (\x_i-\m)]^2,  {I}_j=\{ i = 1,\ldots,n \colon \w_{j}^T (\x_i-\m) \leq 0 \},
%\\[1ex]
%{s}_{2j}= \! \sum\limits_{i \in I_j^c}[ \w_{j}^T (\x_i-\m)]^2, {I}_j^c=\{ i = 1,\ldots,n \colon  \w_{j}^T (\x_i-\m) > 0 \},
%\end{array}
%$$
%and the maximum likelihood estimators of $\sigma_{j}^2$ and $\tau_{j}$ are
%$$\hat \sigma_j^2(\m,W) = \tfrac{1}{n} s_{1j}^{2/3} g_{j}(\m,W), \quad
%\hat \tau_{j}(\m,W)=\left(\frac{s_{2j}}{s_{1j}}\right)^{1/3}.
%$$
%\end{theorem}


Thanks to the above theorem, instead of looking for the maximum of the likelihood function, it is enough to obtain the maximum of the simpler function~(\ref{eq:1}) which depends on two parameters $\m \in \R^d$ and $W \in \M(\R^d)$
\begin{equation}\label{equ:ll}
{l}(X;\m,W) = \prod_{j=1}^{d} {g}_{j}(\m,W)
\end{equation}
where $\w_{j}$ stands for the $j$-th column of matrix $W$. 
Consequently, maximization of (\ref{eq:1}) is equivalent to minimization of  (\ref{equ:ll}), see the following corollary.

\begin{corollary}\label{c2}
Let $X \subset \R^d$, $\m \in \R^d$, $W \in \M(\R^d)$ be given, then
$$
 \text{argmax}_{\m,W} \hat{L}(X;\m,W) =  \argmin_{\m,W} {l}(X;\m,W).
$$
\end{corollary}
%\begin{proof}
%Notice that $l(\m,V,X) = l(\m,W^{-1},X)= {l}(X;\m,W)$ where $W=V^{-1}$ and $$
%\argmax_{\m,V} \hat L(\m,V,X) = \argmin_{\m,V} l(\m,V,X).
%$$
%Moreover
%$$
%\min_{\m,V} l(\m,V,X) = \min_{\m,W} {l}(X;\m,W).
%$$
%
%\end{proof} 


%%%%%%%%%%%%%%%%%%%
\subsection{Gradient}
%%%%%%%%%%%%%%%%%%%

%Minimization of 

%In this subsection we calculate the gradient of the function $\ln({l})$.

One of the possible methods of optimization is the gradient method. Since the minimum of ${l}$ is equal to the minimum of $\ln({l})$, in this subsection we calculate the gradient of $\ln({l})$. 
Before we prove suitable Theorem \ref{ther:grad}, we recall the following lemma. 

\begin{lemma}\label{jacobi}
%\comment{Jacek mowi, ze powino być A(t), i napisać co to znaczy adj()}
Let $A = (a_{ij})_{1 \leq i,j \leq d}$ be a differentiable map from real numbers to $d \times d$ matrices then
\begin{equation}
\frac{\partial \det(A)}{\partial a_{ij}} = \mathrm{adj}^T(A)_{ij},
\end{equation}
where $\mathrm{adj}(A)$ stands for the adjugate of $A$, i.e. the transpose of the cofactor matrix.
\end{lemma}
\begin{proof}
By the Laplace expansion $\det A = \sum\limits_{j=1}^{d} (-1)^{i+j} a_{ij} M_{ij}$ where $M_{ij}$ is the minor of the entry in the $i$-th row and $j$-th column. Hence
$$\frac{\partial \det A}{\partial a_{ij}} = (-1)^{i+j} M_{ij} = \mathrm{adj}^T(A)_{ij}.$$
\end{proof}
Now we are ready to calculate gradient of our cost function.

\begin{theorem}\label{ther:grad}
Let $X \subset \R^d$, $\m = (\m_1, \ldots, \m_d)^T \in \R^d$, $W = (\w_{ij})_{1 \leq i,j \leq d}$ non-singular be given. 
Then
$\nabla_{\m}  \ln {l}(X;\m,W) = \left(  \frac{\partial \ln {l}(X;\m,W)}{\partial \m_1}, \ldots, \frac{\partial \ln {l}(X;\m,W)}{\partial \m_d} \right)^T$,
where
$$
\begin{array}{l}
\frac{\partial \ln {l}(X;\m,W)}{\partial \m_k} =
\sum \limits_{j=1}^d \frac{-1}{{s}_{1j}^{\frac{1}{3}} + {s}_{2j}^{\frac{1}{3}}} \bigg(
\frac{1}{3 {s}_{1j}^{\frac{2}{3}}} \sum \limits_{i \in I_j} 2 \w_j^T (\x_i - \m)  \w_{jk} + %\\[6pt]
\frac{1}{3 {s}_{2j}^{\frac{2}{3}}} \sum \limits_{i \in I_j^c} 2 \w_j^T (\x_i - \m)  \w_{jk}
\bigg).
\end{array}
$$
Moreover,
$
\nabla_{W} \ln {l}(X;\m,W) = \left[ \frac{\partial \ln \tilde{l}(X;\m,W)}{\partial \w_{pk}}  \right]_{1 \leq p,k \leq d},
$
where
$$
\begin{array}{l}
\frac{\partial \ln \tilde{l}(X;\m,W)}{\partial \w_{pk}}  = 
-\frac{2}{3}  (\w^{-1})^T_{pk} +%\\[6pt] 
\frac{1}{{s}_{1p}^{\frac{1}{3}} +{s}_{2p}^{\frac{1}{3}}} 
\bigg(
\frac{1}{3} {s}_{1p}^{-\frac{2}{3}}  \sum \limits_{i \in {I}_p} 2 \w^T_p  (\x_i - \m) (\x_{ik} - \m_k) + \\[6pt]
+ \frac{1}{3} {s}_{2p}^{-\frac{2}{3}}  \sum \limits_{i \in {I}_p^c} 2 \w^T_p  (\x_i - \m) (\x_{ik} - \m_k) \bigg).
\end{array}
$$
and
$$
\begin{array}{c}
%{g}_{j}(\m,W) = {s}_{1j}^{1/3} + {s}_{2j}^{1/3},
%\\[1ex]
{s}_{1j}= \! \sum\limits_{i \in I_j}[ \w_{j}^T (\x_i-\m)]^2, {I}_j=\{ i = 1,\ldots,n \colon \w_{j}^T (\x_i-\m) \leq 0 \},
\\[1ex]
{s}_{2j}= \! \sum\limits_{i \in I_j^c}[ \w_{j}^T (\x_i-\m)]^2,  {I}_j^c=\{ i = 1,\ldots,n \colon  \w_{j}^T (\x_i-\m) > 0 \}.
\end{array}
$$
\end{theorem}

\begin{proof}[Proof of Theorem \ref{ther:grad}.]
Let us start with the partial derivative of $\ln({l})$ with respect to $\m$. We have
$$
\begin{array}{l}
\frac{\partial \ln {l}(X;\m,W)}{\partial \m_k} =
\sum \limits_{j=1}^d \frac{\partial \ln ({g}_j(\m,W))}{\partial \m_k} = \sum\limits_{j=1}^d \frac{1}{{s}_{1j}^{\frac{1}{3}} + {s}_{2j}^{\frac{1}{3}}} \frac{\partial ({s}_{1j}^{\frac{1}{3}} + {s}_{2j}^{\frac{1}{3}})}{\partial \m_k} %=\\[6pt]
 \sum \limits_{j=1}^d \frac{1}{{s}_{1j}^{\frac{1}{3}} + {s}_{2j}^{\frac{1}{3}}} \bigg(
\frac{1}{3 {s}_{1j}^{\frac{2}{3}}} \frac{\partial {s}_{1j}}{\partial \m_k} +
\frac{1}{3 {s}_{2j}^{\frac{2}{3}}} \frac{\partial {s}_{2j}}{\partial \m_k}
\bigg).
\end{array}
$$
Now, we need $\frac{\partial {s}_{1j}}{\partial \m_k}$ and $\frac{\partial {s}_{2j}}{\partial \m_k}$, therefore
$$
\begin{array}{l}
\frac{\partial {s}_{1j}}{\partial \m_k} = 
\sum\limits_{i \in {I}_j} \frac{\partial [\w^T_j (\x_i - \m)]^2}{\partial \m_k} = \sum\limits_{i \in {I}_j} 2 \w^T_j (\x_i - \m) \frac{\partial \w^T_j (\x_i - \m)}{\partial \m_k} = %\\[6pt]
 \sum\limits_{i \in {I}_j} - 2 \w^T_j (\x_i - \m) \w_{jk}.
\end{array}
$$
Analogously we get
$$
\begin{array}{l}
\frac{\partial {s}_{2j}}{\partial \m_k} = \sum\limits_{i \in {I}_j^c} -2 \w^T_j (\x_i - \m) \w_{jk}.
\end{array}
$$
%\comment{$\v^{-1}_{jk} = \w_{jk}$}\\
Hence 
$$
\begin{array}{l}
\frac{\partial \ln {l}}{\partial \m_k} =\sum\limits_{j=1}^d \frac{-1}{{s}_{1j}^{\frac{1}{3}} + {s}_{2j}^{\frac{1}{3}}} \bigg(
\frac{1}{3 {s}_{1j}^{\frac{2}{3}}} \sum\limits_{i \in I_j} 2 \w_j^T (\x_i - \m)  \w_{jk} +% \\[6pt]
\frac{1}{3 {s}_{2j}^{\frac{2}{3}}} \sum\limits_{i \in I_j^c} 2 \w_j^T (\x_i - \m) \w_{jk}
\bigg).
\end{array}
$$

Now we calculate the partial derivative of $\ln {l}(X;\m,W)$ with respect to the matrix $W$. We have
$$
\begin{array}{l}
\frac{\partial \ln {l}(X;\m,W)}{\partial \w_{pk}} = \frac{\partial \ln |\det(W)|^{-\frac{2}{3}}}{\partial \w_{pk}} + \sum\limits_{j=1}^d \frac{\partial \ln ({g}_j(\m,W))}{\partial \w_{pk}}.
\end{array}
$$
%\comment{$\v_{pk}^{-1} = \w_{pk}$}\\
To calculate the derivative of the determinant we use Jacobi's formula (see Lemma \ref{jacobi}).
Hence% $\frac{\partial \ln (\det(W)^{-\frac{2}{3}})}{\partial \w_{pk}} =$
$$
\begin{array}{l}
\frac{\partial \ln (\det(W)^{-\frac{2}{3}})}{\partial \w_{pk}} = \det(W)^{\frac{2}{3}}  \Big(-\frac{2}{3}\Big)  \det(W)^{-\frac{5}{3}}  \frac{\partial \det(W)}{\partial \w_{pk}} = -\frac{2}{3} \det(W)^{-1}  \mathrm{adj}^T(W)_{pk} \\[6pt]
 = -\frac{2}{3} \frac{1}{\det(W)}  \left[\det(W)  (W^{-1})^T_{pk}\right]= -\frac{2}{3}  (\w^{-1})^T_{pk},
\end{array}
$$
where $(\w^{-1})^T_{pk}$ is the element in the $p$-th row and $k$-th column of the matrix $(W^{-1})^T$. Now we calculate %$\frac{\partial \ln ({g}_j(\m,W))}{\partial \w_{pk}} =$
$$
\begin{array}{l}
\frac{\partial \ln ({g}_j(\m,W))}{\partial \w_{pk}} = \frac{1}{{s}_{1j}^{\frac{1}{3}} + {s}_{2j}^{\frac{1}{3}}} \frac{\partial ({s}_{1j}^{\frac{1}{3}} + {s}_{2j}^{\frac{1}{3}})}{\partial \w_{pk}}= \frac{1}{{s}_{1j}^{\frac{1}{3}} + {s}_{2j}^{\frac{1}{3}}} \bigg(
\frac{1}{3 {s}_{1j}^{\frac{2}{3}}}  \frac{\partial {s}_{1j}}{\partial \w_{pk}} +
\frac{1}{3 {s}_{2j}^{\frac{2}{3}}}  \frac{\partial {s}_{2j}}{\partial \w_{pk}}
\bigg),
\end{array}
$$
where
$$
\begin{array}{l}
\frac{\partial {s}_{1j}}{\partial \w_{pk}} = \sum\limits_{ i \in {I}_j} \frac{\partial [\w^T_j (\x_i - \m)]^2}{\partial \w_{pk}} = \sum\limits_{ i \in {I}_j} 2 \w^T_j (\x_i - \m) \frac{\partial \w^T_j (\x_i - \m)}{\partial \w_{pk}}=
\\[6pt]
\left\{ \begin{array}{ll}
0, & \text{if} \; j\neq p\\
\sum\limits_{ i \in {I}_p} 2 \w^T_p (\x_i - \m) (\x_{ik} - \m_k), & \text{if} \; j=p\\
\end{array} \right.
\end{array}
$$
and $\x_{ik}$ is the $k$-th element of the vector $\x_i$. Analogously we get
$$\frac{\partial {s}_{2j}}{\partial \w_{pk}} = \left\{ \begin{array}{ll}
0, & \text{if} \; j\neq p\\
\sum\limits_{ i \in {I}_p^c} 2 \w^T_p (\x_i - \m) (\x_{ik} - \m_k), & \text{if} \; j=p.
\end{array} \right.
$$
Hence we obtain %$\frac{\partial \ln {l}}{\partial \w_{pk}} =$
$$
\begin{array}{l}
\frac{\partial \ln {l}}{\partial \w_{pk}} = -\frac{2}{3} (\w^{-1})^T_{pk} + \frac{1}{{s}_{1p}^{\frac{1}{3}} +{s}_{2p}^{\frac{1}{3}}} 
 \bigg(
\frac{1}{3} {s}_{1p}^{-\frac{2}{3}} \sum\limits_{ i \in {I}_p} 2 \w^T_p (\x_i - \m) (\x_{ik} - \m_k)\\[6pt]
+ \frac{1}{3} {s}_{2p}^{-\frac{2}{3}} \sum\limits_{ i \in {I}_p^c} 2 \w^T_p (\x_i - \m) (\x_{ik} - \m_k) \bigg).
\end{array}
$$
\end{proof}



\section{MODEL II}

\begin{definition}\label{def:GSN}
A density of the multivariate Split Normal $d$ and Normal $D-d$ distribution is given by
$$
 SN_{d}N_{D-d}(\x; \m,W, \sigma^2,\tau^2)=\det(W) \prod_{j=1}^{d} SN(\w_j^T(\x-\m);0,\sigma_j^2,\tau_j^2)\prod_{j=d+1}^{D} N(\w_j^T(\x-\m);0,\sigma_j^2),
$$
where $\w_{j}$ is the $j$-th column of non-singular matrix $W$, $\m = (m_1, \ldots, m_d)^T$, $\sigma = (\sigma_{1},\ldots,\sigma_{d})$ and $\tau=(\tau_{1},\ldots,\tau_{D-d})$.
\end{definition}

%%%%%%%%%%%%%%%%%%%%%%%%%%%%%%%%
%%%%%%%%%%%%%%%%%%%%%%%%%%%%%%%%
 

%% Acknowledgements should only appear in the accepted version. 
%\section*{Acknowledgements} 
% 
%\textbf{Do not} include acknowledgements in the initial version of
%the paper submitted for blind review.
%
%If a paper is accepted, the final camera-ready version can (and
%probably should) include acknowledgements. In this case, please
%place such acknowledgements in an unnumbered section at the
%end of the paper. Typically, this will include thanks to reviewers
%who gave useful comments, to colleagues who contributed to the ideas, 
%and to funding agencies and corporate sponsors that provided financial 
%support.  


% In the unusual situation where you want a paper to appear in the
% references without citing it in the main text, use \nocite
\nocite{langley00}

\bibliography{references}
\bibliographystyle{icml2016}

\end{document} 


% This document was modified from the file originally made available by
% Pat Langley and Andrea Danyluk for ICML-2K. This version was
% created by Lise Getoor and Tobias Scheffer, it was slightly modified  
% from the 2010 version by Thorsten Joachims & Johannes Fuernkranz, 
% slightly modified from the 2009 version by Kiri Wagstaff and 
% Sam Roweis's 2008 version, which is slightly modified from 
% Prasad Tadepalli's 2007 version which is a lightly 
% changed version of the previous year's version by Andrew Moore, 
% which was in turn edited from those of Kristian Kersting and 
% Codrina Lauth. Alex Smola contributed to the algorithmic style files.  

ICA is similar in many aspects to principal component analysis (PCA). In PCA we look for an orthonormal change of basis so that the components are not
linearly dependent (uncorrelated).
ICA can be described as a search for the optimal basis (coordinate system) in which the components are independent. Let us now, for the readers convenience, describe how the 
ICA works. The data are represented by the random vector $\x$
 and the components as the random vector~$s$.  Our aim is to transform the observed data $\x$ into maximally independent components $s$ with respect to some measure  
of independence. Here we use a linear static transformation $W$, called the {\em transformation matrix}, combined with the formula 
$$
s = W \x.
$$

Popular ICA methodology does not directly attempt to find components that are independent but rather components that are as non-Gaussian as possible.
This follows from the fact that one of the theoretical foundations of ICA is given by the dual view at the Central Limit Theorem \cite{hyvarinen2000independent}, which states that the distribution of the sum (average or linear combination) of $N$ independent random variables approaches Gaussian as  $N\rightarrow \infty$. Obviously if all source variables are Gaussian, the ICA method will not work. 

There exists many different approaches to ICA which uses negentropy \cite{hyvarinen2000independent}, cumulant-based methods \cite{cardoso1993blind,virta2015joint}, maximum likelihood methods \cite{chen2006efficient,samworth2012independent} and methods that directly minimize a measure of dependence \cite{stogbauer2004least,matteson2016independent}.

In many application the number of sources is unknown and may be less than the number of sensors. In such situation we are looking for so-called non-square mixing matrix.
In practice, PCA is applied to the observations prior to classic ICA (PCA+ICA) to meet the assumption of square mixing and to reduce computational
costs \cite{hyvarinen2004independent}. PCA+ICA is commonly used to identify brain networks
in functional magnetic resonance imaging (fMRI) \cite{beckmann2012modelling,green2002pca} and hyperspectral unmixing \cite{wang2015abundance,caiafa2008blind}.

The problem in such approach is that interesting independent components (ICs) could be mixed in several principal components that are discarded and then these ICs cannot be recovered.

In the paper we present two methods dedicated to a maximum-likelihood framework. In the firs case we are looking directly $d \leq D$ independent component by maximization of likelihood function. The second method work in full dimensional space by  estimating density congaing $d$ non-gaussian components (independent ones) and $D-d$ gaussian ones which model a noise.  

[!!!Opisac w miare dokladnie nasze podejscie!!!]


